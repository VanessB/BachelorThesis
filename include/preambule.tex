%% Пакеты для работы с математикой.
\usepackage{amsmath,amsfonts,amssymb,amsthm,mathtools}
\usepackage{icomma}
\usepackage{nicematrix} % Красивые матрицы.

%% Работа с русским языком.
\usepackage{cmap}			 % поиск в PDF
%\usepackage{mathtext}		 % русские буквы в формулах
\usepackage[T2A]{fontenc}	 % кодировка
\usepackage[utf8]{inputenc}	 % кодировка исходного текста
\usepackage[russian]{babel}	 % локализация и переносы

%% Номера формул.
\mathtoolsset{showonlyrefs=true} % Показывать номера только у тех формул, на которые есть \eqref{} в тексте.
%\usepackage{leqno}               % Немуреация формул слева

%% Шрифты.
%\usepackage{euscript}	 % Шрифт Евклид
%\usepackage{mathrsfs}   % Красивый матшрифт

%% Поля (геометрия страницы).
\usepackage[left=3cm, right=1.5cm, top=2cm, bottom=2cm, bindingoffset=0cm]{geometry}
%headheight=28pt

%% Русские списки.
\usepackage{enumitem}
\makeatletter
\AddEnumerateCounter{\asbuk}{\russian@alph}{щ}
\makeatother

%% Работа с картинками.
\usepackage{caption}           % Пакет для подписей к рисункам, в частности, для работы caption*.
%\usepackage{subcaption}        % Подкартинки.

\captionsetup{justification=centering} % Центрирование подписей к картинкам.
\usepackage{graphicx}                  % Для вставки рисунков.
\graphicspath{{images/}{images2/}}     % Папки с картинками.
\setlength\fboxsep{3pt}                % Отступ рамки \fbox{} от рисунка.
\setlength\fboxrule{1pt}               % Толщина линий рамки \fbox{}.
\usepackage{wrapfig}                   % Обтекание рисунков и таблиц текстом.

%% Работа с таблицами.
%\usepackage{array,tabularx,tabulary,booktabs} % Дополнительная работа с таблицами.
%\usepackage{longtable}                        % Длинные таблицы.
%\usepackage{multirow}                         % Слияние строк в таблице.

%% Интервалы.
\linespread{1.5}              % Междустрочный интервал.
\setlength\parindent{1.25cm}  % Абзацный отступ.

%% TikZ.
%\usepackage{tikz}
%\usetikzlibrary{graphs,graphs.standard}

%% Графики gnuplot.
\usepackage[shell, subfolder, cleanup]{gnuplottex}

%% Верхний колонтитул.
%\usepackage{fancyhdr}
%\pagestyle{fancy}

%% Перенос знаков в формулах (по Львовскому).
\newcommand*{\hm}[1]{#1\nobreak\discretionary{}{\hbox{$\mathsurround=0pt #1$}}{}}

%% Дополнения.
\usepackage{float}             % Добавляет возможность работы с командой [H] которая улучшает расположение на странице.
\usepackage{textcomp, gensymb} % Красивые градусы.

%% Вращение pdf или его наполнения.
\usepackage{rotating}  % Вращение плавающих объектов.
\usepackage{pdflscape} % Вращение страницы.

%% Объекты на одтельной странице.
%\usepackage{afterpage}

%% Контроль положения плавающих объектов.
\usepackage{placeins}

%% hyperref для ссылок внутри  pdf.
\usepackage[unicode, pdftex]{hyperref}

%% Список литературы.
\usepackage[backend=biber, style=gost-numeric]{biblatex}
\usepackage{csquotes}

%% Стили заголовков.
\usepackage{titlesec}
% Глава.
\titleformat{\chapter}{\normalfont\fontsize{14}{15}\bfseries}{\chaptertitlename\ \thechapter:}{1em}{}
\titlespacing*{\chapter}{0pt}{0pt}{10pt}
% Секция.
\titleformat{\section}{\normalfont\fontsize{14}{15}\bfseries}{\thesection}{1em}{}
% Подсекция.
\titleformat{\subsection}{\normalfont\fontsize{14}{15}\bfseries}{\thesubsection}{1em}{}
