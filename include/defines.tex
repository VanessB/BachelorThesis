%% Множества.
\def\NN{\mathbb{N}} % Натуральные числа.
\def\ZZ{\mathbb{Z}} % Целые числа.
\def\RR{\mathbb{R}} % Действительные числа
\def\CC{\mathbb{C}} % Комплексные числа.

%% Линейная алгебра.
\def\diag{\operatorname{diag}} % Диагональная матрица.

%% Математический анализ.
\newcommand{\limto}[2]{\underset{#1 \to #2}{\longrightarrow}}  % Стрелка, обозначающая предел.
\newcommand{\overlimto}[3]{\overset{#3}{\limto{#1}{#2}}}       % Стрелка, обозначающая предел, с подписью типа сходимости.
\def\banachspace{\mathcal{B}}                                  % Банахово пространство.
\def\hilbertspace{\mathcal{H}}                                 % Гильбертово пространство.
\newcommand{\dotprod}[2]{\left\langle #1, #2 \right\rangle}    % Скалярное произведение.

%% Комплексные числа.
\def\Re{\operatorname{Re}} % Действительная часть.
\def\Im{\operatorname{Im}} % Мнимая часть.

%% Численные методы.
\def\res{\mathcal{R}}              % Невязка
\def\jac{\mathcal{J}}              % Матрица Якоби
\newcommand{\bvec}[1]{\mathbf{#1}} % ``Жирный'' вектор.
\def\stabreg{\mathbf{R}}           % Область устойчивости.
\def\niter{N_{\text{итер}}}        % Число итераций.
\def\abseps{\varepsilon_{\text{абс}}} % Абсолютная погрешность.
\def\releps{\varepsilon_{\text{отн}}} % Относительная погрешность.

\newcommand{\lognorm}[1]{\mu \left[ #1 \right]} % Логарифмическая норма.

\def\taulin{\tau_{\textnormal{lin}}}       % Характерное время линейной реакции системы на возмущения.
\def\taunonlin{\tau_{\textnormal{nonlin}}} % Характерное время, за которое меняется матрица Якоби правой части системы.

%% Асимптотические классы.
\def\Oclass{\mathcal{O}} % О-большое.

%% Выделение определения.
\DeclareTextFontCommand{\defemph}{\bfseries\em}
