\chapter{Перспективы клинического использования}
\label{chapter:perspectives} \index{Перспективы}

При исследовании больных с пороками сердца при помощи методов эхокардиографии многие исследователи обнаруживают трмбы в
полостях правого и левого сердца.
При этом тромбы чаще располагаются в левом сердце, а именно в ушке левого предсердия, чем в правом сердце
\cite{ikorkin2018diagnosis, agmon2002clinical, castello1990prevalence, pollick1991assesmentlaa}.

Придаток левого предсердия имеет характерную изогнутую S-образную форму, форму гребня или полуовала,
покрыт изнутри гребенчатыми мышцами и может содержать круговые мышечные волокна \cite{stepanchuk2012appendage}.
Пример придатка в здоровом сердце приведен на рисунке~\ref{fig:normal_laa}.
При патологиях в сердце (мерцательная аритмия, ишемия) мышцы атрофируются и замещаются соединительной тканью,
характер течения в придатке меняется, повышается риск образования в нем тромба.
Пример придатка в сердце с паталогией приведен на рисунке~\ref{fig:pathologic_laa}.

\begin{figure}[ht!]
    \begin{center}
        \raisebox{-0.5\height}{\includegraphics[width=0.45\linewidth]{left_atrial_appendage/pic_1.png}}\hspace{8pt}
        \raisebox{-0.5\height}{\includegraphics[width=0.45\linewidth]{left_atrial_appendage/pic_2.png}}
    \end{center}
    \caption{Внешний вид левого ушка (сердца без патологии) \cite{stepanchuk2012appendage}.
    1~--- верхний край, 2~--- нижний край с дольчатыми придатками, 3~--- верхушка, 4~--- шейка, 5~--- тело.}
    \label{fig:normal_laa}
\end{figure}


\begin{figure}[ht!]
    \begin{center}
        \raisebox{-0.5\height}{\includegraphics[width=0.45\linewidth]{left_atrial_appendage/pic_5.png}}\hspace{8pt}
        \raisebox{-0.5\height}{\includegraphics[width=0.45\linewidth]{left_atrial_appendage/pic_6.png}}
    \end{center}
    \caption{Внешний (А) и внутренний (Б) вид левого ушка при митральном пороке \cite{stepanchuk2012appendage}.
    1~--- верхний край, 2~--- нижний край с куполоподобными полостями, 3~--- устье, 4~--- тело, 5~--- верхушка.}
    \label{fig:pathologic_laa}
\end{figure}

При мерцательной аритмии в ушке начинается активное образование тромбов,
что грозит их отрывом и последующим инсультом.
Данный процесс купируется при помощи антикоагулянтов,
однако их избыток нарушает естественную свертываемость крови, что нежелательно.
В дальнейшем планируется применить математическое моделирование для определения
вероятности образования тромба при заданной концентрации антикоагулянта.
Для этого требуется моделировать течение крови и систему переноса-диффузии-реакции,
описанные в \cite{vassilevski2020parallel}, в геометрии ушка.
При этом, как было показано ранее, используемая система реакций свёртывания крови \eqref{eq:blood_coagulation_cascade} является жесткой.
Также может требоватся подстройка под условия лечения (изменение формул для коагулянта и различных факторов).
Поэтому необходимо иметь устойчивый неявный метод интегрирования систем реакций,
допускающий при этом возможность настройки последней без вреда для своих полезных свойств.

Предложенные в данной работе численные методы планируется внедрить в программный пакет INMOST \cite{vassilevski2020parallel}
в виде специального модуля для решения реакционной части систем.
Для этого в том числе необходимо программно реализовать алгоритм вычисления матричных функций.
Как уже было упомянуто в разделе \ref{sec:modified_Newton},
для этой цели предлагается использовать спектральное разложение матрицы.
В качестве алгоритма поиска спектра матрицы выбран \emph{алгоритм Франциса},
описанный в работах \cite{francis1961first, francis1962second}.
Выбранный алгоритм позволяет выполнять QR-итерацию с произвольными сдвигами (в том числе и комплексными),
оставаясь при этом в вещественной арифметике.
Данное свойство является существенным преимуществом,
так как оно заметно уменьшает размер требуемой для итераций памяти и число операций над числами с плавающей точкой.
