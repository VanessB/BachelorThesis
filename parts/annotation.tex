\begin{abstract}

    \begin{center}
        \large{Решение жестких систем реакций свертывания крови и моделирование образования тромба в придатке левого желудочка} \\
    \large\textit{Бутаков Иван Дмитриевич} \\[1 cm]
    \end{center}

    При патологиях в сердце характер течения в придатке левого предсердия меняется, повышается риск образования в нем тромба.
    Для моделирования процесса образования тромба требуется решать систему переноса-диффузии-реакции, где реакционная часть представлена жёсткой системой каскада свёртывания крови.
    Применение традиционных численных схем при интегрировании данной системы может вести к неустойчивости, нефизичным осцилляциям,
    к отсутствию сходимости итерационных методов решения возникающих нелинейных уравнений.
    В данной работе предложены два метода для неявного численного интегрирования жёстких нелинейных систем, способные до некоторой степени решить указанные проблемы.
    Методы основаны на <<сглаживании>> спектра матрицы Якоби правой части системы.
    Сглаживание производится путём комбинирования явного и неявного метода Эйлера с матричным весом.
    Весовая матрица вычисляется путём применения специально подобранной функции к спектру матрицы Якоби правой части.
    Подбор функции осуществлён с целью получения экспоненциального интегратора.
    В ходе работы полученные методы были проверены на следующих жёстких системах: модель Лотки-Вольтерры, модель осциллятора Ван-дер-Поля, модель каскада свёртывания крови.
    Во всех случаях предложенные методы показали улучшение устойчивости в сравнении с невзвешенным вариантом.
    В работе также описаны детали программной реализации предложенных методов.
    Наконец, дан краткий обзор возможностей по применению предложенной техники для построения других методов интегрирования, обладающих схожими свойствами.

    \vfill

    \textbf{Abstract} \\[1 cm]

    Solving stiff blood coagulation system and modeling clot formation in left atrial appendage
    %\end{center}

\end{abstract}
\newpage
