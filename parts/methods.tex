\chapter{Разработка численных методов}
\label{chapter:methods} \index{Методы}

В прошлой главе был дан краткий обзор существующих методов оценки устойчивости динамических систем и численных схем.
Введены основные элементы линейного и нелинейного анализа устойчивости.
Введено понятие жёсткости системы, разделённое в дальнейшем на линейную и нелинейную жёсткость.
Было показано, что с линейной жёсткостью эффективно справляются A-устойчивые и L-устойчивые методы,
в том числе и экспоненциальные интеграторы, интегрирующие линейную часть системы точно.
Наконец, обозначена невозможность решения проблем нелинейной жёсткости в рамках линейной теории устойчивости.

Данный раздел полностью посвящён разработке новых методов численного решения дифференциальных уравнений,
потенциально способных показать более устойчивое поведение на нелинейно жёстких системах.
В то же время, разрабатываемые методы должны устойчиво интегрировать линейно жёсткие системы.

Первый раздел данной главы посвящён разработке квазиньютоновской модификации стандартной неявной схемы Эйлера.
Данная модификация основана на фильтрации спектра матрицы Якоби правой части системы.
Во втором разделе рассматривается численная схема, являющаяся взвешенной комбинацией явного и неявного метода Эйлера,
и дающая матрицу Якоби, совпадающую с модифицированной матрицей из первого раздела.



\section{Модифицированный метод Ньютона}
\label{sec:modified_Newton}

Рассмотрим неявный метод Эйлера, невязка и матрица Якоби невязки которого даны в \eqref{eq:implicit_Euler_residual} и \eqref{eq:implicit_Euler_jacobian} соответственно.
Для получения значения $ \bvec{x} $ на следующем, $ (n + 1) $-ом шаге по времени,
будем решать в общем случае нелинейное уравнение $ \res(\bvec{x}^{n+1}) = \bvec{0} $ методом Ньютона:
%
\begin{equation}
    \label{eq:Newton_method}
    \jac_m \cdot \left( \bvec{x}_{m+1}^{n+1} - \bvec{x}_m^{n+1} \right) = - \res_m
    \qquad \Longrightarrow \qquad
    \bvec{x}_{m+1}^{n+1} = \bvec{x}_m^{n+1} - \jac_m^{-1} \cdot \res_m,
\end{equation}
%
где $ m $~--- номер нелинейной итерации метода, и используются обозначения
%
\begin{equation}
    \label{eq:Newton_method_defines}
    \res_m = \res(t^{n+1}, \bvec{x}_m^{n+1}),
    \quad \jac_m = \jac(t^{n+1}, \bvec{x}_m^{n+1}) = I - \Delta t \cdot F_m,
\end{equation}
\begin{equation}
    \label{eq:Newton_method_defines_2}
    \quad F_m = F(t^{n+1}, \bvec{x}_m^{n+1}) = \frac{\partial f}{\partial \bvec{x}}(t^{n+1}, \bvec{x}_m^{n+1})
\end{equation}

Если $ F_m $ имеет седловую структуру, то для достаточно большого $ \Delta t $ матрица $ \jac_m $ не знакоопределена.
Рассмотрим функцию $ \theta(z) $ комплексного аргумента, регулярную в такой области, которая на любой итерации содержит $ \Delta t \cdot \sigma(F_m) $.
Тогда на каждом шаге определена матрица $ M = M_m = \theta(\Delta t \cdot F_m) $.
Модифицируем шаг метода Ньютона следующим образом:
%
\begin{equation}
    \label{eq:modified_Newton_iteration}
    \bvec{x}_{m+1}^{n+1} = \bvec{x}_m^{n+1} - (I - \Delta t \cdot M F_m)^{-1} \cdot \res_m,
\end{equation}
%
В зависимости от выбора $ \theta(z) $ весовая матрица $ M $ позволяет <<фильтровать>> спектр $ F_m $
и регулировать его влияние на модифицированную матрицу Якоби $ \bar{\jac_m} = I - \Delta t \cdot M F_m $.
Введём следующий класс функций:

\begin{definition}
    \label{def:generalized_sign_function}
    Функцию $ \varphi(z) $ назовём \emph{обобщённой функцией знака} в случае, если
    \[
        \lim_{\Re z \to \pm \infty} \varphi(z) = \pm 1
    \]
\end{definition}

Если рассматривать $ \theta(z) = \frac{1}{2} (1 - \varphi(z)) $,
то с ростом $ \Delta t $ можно ожидать ослабления положительной действительной части спектра $ \Delta t \cdot F_m $ при умножении на $ M $.
При этом часть спектра с отрицательной действительной частью должна фильтроваться в меньшей степени.
Это приближает матрицу $ \bar{\jac}_m = I - \Delta t \cdot M F_m $ к отрицательно определённой,
что потенциально может улучшить глобальную сходимость модифицированного метода Ньютона.
Выбор конкретной функции $ \theta(z) $ из представленного семейства будет рассмотрен в следующих секциях.

Затронем вопрос вычисления матрицы $ M $.
Диагонализуемые матрицы плотны в множестве квадратных матриц.
Соответственно, если $ f(t, \bvec{x}) $ не обладает особыми свойствами, логично ожидать, что, как правило,
матрица Якоби правой части будет диагонализуема.
Пусть $ F = V \Lambda V^{-1} $, $ \Lambda = \diag(\lambda_i) $.
Тогда, согласно \ref{rem:regular_function_diagonalizable_operator}, $ M = \theta(F) = V \theta(\Lambda) V^{-1} = V \diag\left( \theta(\lambda_i) \right) V^{-1} $,
что уже легко вычисляется, если известны $ V $ и $ \Lambda $.
При определённых условиях на $ \theta $ матрица $ M $ всегда будет вещественной.
Получим эти условия в виде достаточного признака:

\begin{statement}
    \label{stat:whole_real_function}
    Пусть $ f: \CC \to \CC $~--- целая (регулярная во всей $ \CC $) функция.
    Она принимает на $ \RR $ только вещественные значения тогда и только тогда,
    когда имеет вещественные коэффициенты в разложении в ряд Тейлора в любой точке $ \RR $.
\end{statement}

\begin{proof}
    Докажем в обе стороны.
    \begin{itemize}
        \item[$ \Rightarrow $]
            Пусть $ f $ принимает только вещественные значения на $ \RR $.
            Поскольку она целая, её можно представить в виде сходящегося ряда Тейлора,
            записанного относительно произвольной точки.
            Пусть $ x_0 \in \RR $.
            Тогда
            \[
                f(x) = \sum_{k=0}^{\infty} \frac{f^{(k)}(x_0)}{k!} (x - x_0)^k
            \]
            Поскольку $ f(x) $ принимает только вещественные значения на $ \RR $,
            все её производные обладают тем же свойством.
            Отсюда получаем вещественность коэффициентов в разложении в ряд Тейлора.
        \item[$ \Leftarrow $]
            Пусть $ f $ имеет вещественные коэффициенты в разложении в ряд Тейлора относительно хотя бы одной точки $ x_0 \in \RR $.
            Тогда
            \[
                \forall x \in \RR \quad f(x) = \sum_{k=0}^{\infty} c_k (x - x_0)^k,
            \]
            где $ c_k \in \RR $ для любого $ k $.
            Значит, значение $ f(x) $ совпадает со значением суммы ряда вещественных чисел,
            а потому вещественно.
    \end{itemize}
\end{proof}

\begin{lemma}
    \label{lem:whole_real_function_matrix}
    Пусть $ f: \CC \to \CC $~--- целая функция, принимающая на $ \RR $ только вещественные значения.
    Пусть $ A \in \RR^{d \times d} $.
    Тогда $ f(A) $~--- также вещественная матрица.
\end{lemma}

\begin{proof}
    Поскольку $ f $~--- целая,
    \[
        f(A) = \sum_{k=0}^{\infty} c_k \cdot A^k
    \]
    Из утверждения \ref{stat:whole_real_function} имеем, что $ \forall k \in \NN_0 \;\, c_k \in \RR $.
    Тогда матрица $ f(A) $ равна сумме ряда вещественных матриц, а потому вещественна.
\end{proof}

\begin{corollary}
    \label{cor:real_matrix}
    Пусть $ \theta(z) $ является результатом применения стандартных алгебраических операций (сложение, умножение, обращение)
    к значениям $ f_1(z), \ldots, f_N(z) $, где $ f_i $~--- целая функция, принимающая на $ \RR $ только вещественные значения.
    Пусть $ A \in \RR^{d \times d} $.
    Тогда $ \theta(A) $~--- также вещественная матрица.
\end{corollary}



\section{Взвешенный метод Эйлера}
\label{sec:weighted_Euler}

Предложенная в предыдущем разделе модификация является квазиньютоновской.
Её побочным эффектом является ухудшение скорости сходимости при поиске корня невязки из-за неточного направления в методе Ньютона.
В данной секции вводится численная схема, дающая невязку с той же матрицей Якоби,
что получается в предыдущей секции.
Рассмотрим модификацию невязки
%
\begin{multline}
    \label{eq:modified_residual}
    \bar{\res}(t^{n+1}, \bvec{x}^{n+1}) = M \cdot \res(t^{n+1}, \bvec{x}^{n+1}) + (I - M) \cdot \left( \bvec{x}^{n+1} - \bvec{x}^n - \Delta t \cdot f(t^n, \bvec{x}^n) \right) = \\
    = \bvec{x}^{n+1} - \bvec{x}^n - \Delta t \cdot \left[ M f(t^{n+1}, \bvec{x}^{n+1}) + (I - M) f(t^n, \bvec{x}^n) \right],
\end{multline}
%
где $ \res $ взята из \eqref{eq:implicit_Euler_residual}.
Поскольку суммарный коэффициент при $ f $ равен $ M + (I - M) = I $,
предложенная схема имеет, как минимум, первый порядок аппроксимации.

Несложно заметить, что при фиксированной $ M $ итерация метода Ньютона принимает вид,
аналогичный \eqref{eq:modified_Newton_iteration}:
%
\begin{equation}
    \label{eq:weighted_Newton_iteration}
    \bvec{x}_{m+1}^{n+1} = \bvec{x}_m^{n+1} - (I - \Delta t \cdot M F_m)^{-1} \cdot \bar{\res}_m,
\end{equation}
%
Выпишем полностью алгоритм Ньютона для нахождения корней невязки полученной схемы:
\begin{itemize}
    \item Выбрать начальное приближения $ \bvec{x}^{n+1}_0 = \bvec{x}^n $ и вычислить $ f^n = f(t^n, \bvec{x}^n) $.
    \item Выполнить $ m $-ую итерацию метода Ньютона:
        \begin{itemize}
            \item
                Вычислить значение функции $ f_m = f(t^{n+1}, \bvec{x}_m^{n+1}) $ и матрицу производных
                $ F_m = \frac{\partial f}{\partial \bvec{x}}(t^{n+1}, \bvec{x}_m^{n+1}) $.
            \item Вычислить собственное разложение матрицы $ F_m = V \Lambda V^{-1} $.
            \item Вычислить $ \theta(F_m) = V \theta(\Lambda) V^{-1} $ и $ \bar{\res}_m = \bvec{x}_m^{n+1} - \bvec{x}^n - \Delta t \cdot \left[ M f_m^{n+1} + (I - M) f^n \right] $.
            \item
                Если $ \| \bar{\res}_m \| > \max \left\{ \abseps, \releps \cdot \| \bar{\res}_0 \| \right\} $,
                то перейти на следующую итерацию по формуле \eqref{eq:weighted_Newton_iteration},
                иначе $ \bvec{x}^{n+1} = \bvec{x}_m^{n+1} $ и завершить выполнение.
        \end{itemize}
\end{itemize}



\section{Выбор весовой функции}
\label{sec:choosing_weight}

В предыдущих двух разделах были введены два семейства численных методов,
основанных на <<фильтрации>> спектра матрицы Якоби правой части системы
посредством её умножения на матрицу $ M $,
получаемой применением весовой функции $ \theta(z) $ к $ \Delta t \cdot F $.
В данном разделе будет приведён один из возможных вариантов выбора $ \theta(z) $.

Как уже отмечалось в начале настоящей главы,
задача построения новых численных методов в данной работе заключается в решении проблемы нелинейной жёсткости.
Однако также было упомянуто, что предложенные методы должны удовлетворительно справляться и с эффектами линейной жёсткости.
Метод \ref{sec:modified_Newton} обладает той же функцией устойчивости, что и неявный метод Эйлера,
а потому уже A-устойчив независимо от выбора $ \theta(z) $.
С другой стороны, метод \ref{sec:weighted_Euler} обладает функцией устойчивости,
указанной в \eqref{eq:weighted_two_point}.
В общем случае она может не содержать $ \CC^- $.

Выберем $ \theta(z) $ так, чтобы взвешенный метод Эйлера точно интегрировал линейную часть системы,
то есть чтобы получился экспоненциальный интегратор.
Из \eqref{eq:weighted_two_point} и замечания \ref{rem:exponential_integrator_stability_function} имеем
\[
    R(z) = \frac{1 + (1 - \theta(z)) z}{1 - \theta(z) z} = e^z
\]
Приводя систему, получим
\[
    e^z (1 - \theta(z) z) = 1 + (1 - \theta(z)) z,
\]
откуда
\[
    \theta(z) \left( 1 - e^z \right) z = 1 + z - e^z
\]
В результате имеем
\begin{equation}
    \label{eq:exponential_integrator_weight}
    \theta(z) =
    \begin{dcases}
        \frac{1}{z} - \frac{1}{e^z - 1}, &\quad z \neq 2 \pi i \cdot k, \; k \in \ZZ \\
        \frac{1}{2}, &\quad z = 0,
    \end{dcases}
\end{equation}
где функция сразу доопределена в нуле своим пределом.
Она регулярна на всей области определения.
Заметим также, что $ \theta(z) = \frac{1}{2} (1 - \varphi(z)) $, где
\begin{equation}
    \label{eq:exponential_integrator_sign}
    \varphi(z) =
    \begin{dcases}
        1 - \frac{2}{z} + \frac{2}{e^z - 1}, &\quad z \neq 2 \pi i \cdot k, \; k \in \ZZ \\
        0, &\quad z = 0
    \end{dcases}
\end{equation}
--- обобщённая функция знака по определению \ref{def:generalized_sign_function}.
Более того,
\[
    \forall z \in \CC \quad 1 - z \theta(z) = \frac{z}{e^z - 1} \neq 0,
\]
что гарантирует невырожденность матрицы $ \bar{\jac} $.
Также выбранная весовая функция попадает под условия следствия \ref{cor:real_matrix},
что гарантирует вещественность $ \theta(\Delta t \cdot F) $ для вещественных $ F $.
Наконец,
\begin{equation}
    \label{eq:exponential_integrator_weight_series}
    \theta(z) = \frac{1}{2} - \frac{z}{12} + \frac{z^3}{720} + O(z^4)
\end{equation}
Таким образом, при $ \Delta t \to 0 $ взвешенный метод Эйлера с данной весовой функцией переходит в метод трапеций.
