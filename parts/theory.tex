\chapter{Теоретические сведения}
\label{chapter:theory} \index{Теория}

Прежде чем перейти к описанию численных методов, остановимся подробнее на теории жёстких систем дифференциальных уравнений.
В данном разделе приведены
%необходимые для дальнейшего анализа
элементы линейной и нелинейной теории устойчивости численных методов.
Отметим, что данная теория гораздо более полно описана в работах \cite{heirer1999solvingode2, lambert1991methods}.

В дальнейшем часто будет рассматриваться задача Коши вида
%
\begin{equation}
    \label{eq:initial_value_problem}
    \begin{dcases}
        \frac{d \bvec{x}}{d t} = f(t, \bvec{x}), \\
        \bvec{x}(0) = \bvec{x}_0,
    \end{dcases}
    \qquad
    \bvec{x} \in \RR^d, \qquad f: [0;T] \times D \to \RR^d,
\end{equation}
%
где $ D \subseteq \RR^d $.
Поскольку любую систему обыкновенных дифференциальных уравнений можно свести к автономной,
%добавив уравнение $ dt / dt = 1 $,
запись задачи можно упростить:
%
\begin{equation}
    \label{eq:autonomous_initial_value_problem}
    \begin{dcases}
        \frac{d \bvec{x}}{d t} = f(\bvec{x}), \\
        \bvec{x}(0) = \bvec{x}_0
    \end{dcases}
\end{equation}
%
Раскладывая $ f(\bvec{x}) $ в ряд Тейлора в окрестности $ \bvec{x}_0 $ и пренебрегая членами, содержащими производные порядка выше второго, можно получить линеаризованную задачу Коши:
%
\begin{equation}
    \label{eq:linearized_initial_value_problem}
    \begin{dcases}
        \frac{d \bvec{x}}{d t} = f_0 + F_0 \cdot (\bvec{x} - \bvec{x}_0), \\
        \bvec{x}(0) = \bvec{x}_0,
    \end{dcases}
\end{equation}
%
где $ \left. \frac{\partial f}{\partial \bvec{x}} \right|_{\bvec{x}_0} = F(\bvec{x}_0) \equiv F_0 $~--- матрица Якоби правой части уравнения в точке $ \bvec{x}_0 $.
Для линеаризованной задачи известно точное решение:
\begin{equation}
    \label{eq:linearized_solution}
    \bvec{x}(t) = \bvec{x}_0 + (\exp(t \cdot F_0) - I) F_0^{-1} \cdot f_0
\end{equation}
Стоит отметить, что выражение $ (\exp(A) - I) A^{-1} $ можно определить для всех матриц, так как
\begin{equation}
    \label{eq:phi_1_function}
    \varphi_1(z) =
    \begin{dcases}
        \frac{e^z - 1}{z}, &\quad z \neq 0 \\
        1, &\quad z = 0
    \end{dcases}
\end{equation}
--- регулярная в $ \CC $ функция.
Для этого введём следующее определение:

\begin{definition}
    \label{def:regular_function_operator}
    Пусть $ A $~--- линейный ограниченный оператор, действующий в банаховом пространстве $ \banachspace $ над $ \CC $,
    $ \sigma(A) $~--- спектр оператора $ A $, а
    $ f $~--- регулярная в области $ U \supset \sigma(A) $ функция.
    Тогда можно определить
    \begin{equation}
        \label{eq:regular_function_operator}
        f(A) = \frac{1}{2 \pi i} \int\limits_{\Gamma} f(\xi) \left( \xi I - A \right)^{-1} d \xi,
    \end{equation}
    где $ \Gamma $~--- произвольный гладкий контур в $ U $ такой,
    что $ \sigma(A) $ целиком находится по левую сторону при положительном обходе $ \Gamma $.
\end{definition}

Корректность данного определения доказана в \cite{takesaki2001opalgebras1}.
Также справедливо следующее замечание:

\begin{remark}
    \label{rem:regular_function_diagonalizable_operator}
    Пусть в условиях опредления \ref{def:regular_function_operator} оператор $ A $ диагонализуемый: $ A = V \Lambda V^{-1} $, $ \Lambda = \diag(\lambda_{\alpha}) $.
    Тогда $ f(A) = V f(\Lambda) V^{-1} = V \diag \left(f(\lambda_{\alpha}) \right) V^{-1} $.
\end{remark}



\section{Линейная теория устойчивости}
\label{sec:linear_stability}

При численном решении систем дифференциальных уравнений неизбежны неточности и численные погрешности, возмущения.
Поведение систем и численных методов под действием данных возмущений во многом определяет возможность получения точного численного решения,
а также вычислительные ресурсы, которые для этого потребуются.
Линейная теория устойчивости численных методов подходит к данному вопросу со стороны линеаризации динамической системы в окрестности некоторой точки или решения.
В рамках данной теории рассматривается поведение численного метода при решении уравнения \eqref{eq:linearized_initial_value_problem}.
В случае невырожденной $ F_0 $ при помощи линейной замены $ \bvec{x} := \bvec{x} - \bvec{x}_0 + F_0^{-1} f_0 $ задача сводится к проверочному \emph{уравнению Далквиста} \cite{dahlquist1963special}:
%
\begin{equation}
    \label{eq:Dahlquist_equation}
    \begin{dcases}
        \dfrac{d \bvec{x}}{d t} = F_0 \cdot \bvec{x}, \\
        \bvec{x}(0) = \bvec{x}_0
    \end{dcases}
    \qquad
    \Longleftrightarrow
    \qquad
    \bvec{x}(t) = \exp(t \cdot F_0) \cdot \bvec{x}_0
\end{equation}

Для линейных (возможно, многошаговых) численных схем в случае $ f(\bvec{x}) = F_0 \cdot \bvec{x} $ один шаг можно записать в виде
%
\begin{equation}
    \label{eq:stability_function}
    \bvec{x}^{n+1} = R(\Delta t \cdot F_0) \cdot \bvec{x}^n,
\end{equation}
%
где $ R(z) $~--- \emph{функция устойчивости}, а $ R(\Delta t \cdot F_0) $~--- \emph{матрица перехода}.
Обычно $ R(z) $ рассматривается как функция комплексного переменного, и потом естественным образом обобщается на матричный аргумент
(см. \ref{def:regular_function_operator}).
По умолчанию далее так и будет подразумеваться.
Однако в силу специфики данной работы, понятие функции устойчивости будет иногда обобщаться, сразу предполагая матричный характер
(то есть прообраз среди функций комплексного переменного может отсутствовать).

Приведём несколько примеров линейных численных схем и их функций устойчивости.
Эти примеры понадобятся при дальнейшем анализе.
%
\begin{align}
    \text{\emph{Явный метод Эйлера:}}   && \frac{\bvec{x}^{n+1} - \bvec{x}^n}{\Delta t} &= f(\bvec{x}^n) & R(z) &= 1 + z & \label{eq:forward_Euler} \\
    \text{\emph{Неявный метод Эйлера:}} && \frac{\bvec{x}^{n+1} - \bvec{x}^n}{\Delta t} &= f(\bvec{x}^{n+1}) & R(z) &= \frac{1}{1 - z} & \label{eq:backward_Euler} \\
    \text{\emph{Метод трапеций:}}       && \frac{\bvec{x}^{n+1} - \bvec{x}^n}{\Delta t} &= \frac{1}{2} f(\bvec{x}^n) + \frac{1}{2} f(\bvec{x}^{n+1}) & R(z) &= \frac{2 + z}{2 - z} & \label{eq:trapezoid}
\end{align}
%
В общем случае двухшаговой одностадийной схемы:
\begin{equation}
    \frac{\bvec{x}^{n+1} - \bvec{x}^n}{\Delta t} = (1 - M) f(\bvec{x}^n) + M f(\bvec{x}^{n+1}) \qquad R(z) = (1 - M z)^{-1}(1 + (1 - M) z), \label{eq:weighted_two_point}
\end{equation}
где $ M $, вообще говоря, может быть матрицей (в таком случае $ R(\cdot) $ следует изначально рассматривать как матричную функцию,
которая, однако, при определённом выборе $ M $ может иметь аналог среди функций комплексного переменного%
\footnote{В частности, далее будет показано, что выбором $ M $ можно добиться $ R(A) = \exp(A) $,
из чего следует $ R(z) = e^z $.}).
Если же $ M $~--- число, то данная схема называется \emph{$ \theta $-методом}.


\subsection{Линейная устойчивость}
\label{subsec:linear_stability}

Рассмотрим формально вопросы затухания аналитического и численного решения с течением времени.
Для этого потребуются следующие определения и теоремы:

\begin{definition}
    \label{def:spectral_radius_and_abscissa}
    Пусть $ \sigma(A) \subseteq \CC $~--- спектр линейного оператора $ A $.
    Число $ \displaystyle r(A) = \sup \{|\lambda| \mid \lambda \in \sigma(A) \} $ называется \emph{спектральным радиусом} линейного оператора $ A $,
    а $ \displaystyle s(A) = \sup \{\Re \lambda \mid \lambda \in \sigma(A) \} $~--- \emph{спектральной границей}.
\end{definition}

\begin{theorem}[об отображении спектра; \cite{takesaki2001opalgebras1}, утверждение 2.8]
    \label{thm:spectral_mapping_theorem}
    Пусть $ A $~--- линейный ограниченный оператор, действующий в банаховом пространстве $ \banachspace $ над $ \CC $,
    $ f $~--- регулярная в области $ U \supset \sigma(A) $ функция.
    Тогда
    \begin{equation}
        \label{eq:spectral_mapping_theorem}
        \sigma(f(A)) = f(\sigma(A))
    \end{equation}
\end{theorem}

Здесь и далее под нормой оператора будет подразумеваться норма,
подчинённая норме линейного пространства,
в котором действует оператор: $\displaystyle \| A \| = \sup_{\| x \| = 1} \| A x \| $.

\begin{theorem}[формула Бёрлинга-Гельфанда]
    \label{thm:Beurling-Gelfand_formula}
    Пусть $ A $~--- линейный ограниченный оператор, действующий в банаховом пространстве $ \banachspace $ над $ \CC $.
    Тогда $ \displaystyle r(A) = \lim_{n \to \infty} \| A^n \|^{\frac{1}{n}} = \inf_{n \in \NN} \| A^n \|^{\frac{1}{n}} $.
\end{theorem}

\begin{corollary}
    \label{cor:spectral_radius_norm_bounds}
    Пусть $ A $~--- линейный ограниченный оператор, действующий в банаховом пространстве $ \banachspace $ над $ \CC $.
    Тогда
    \[
        \forall \varepsilon > 0 \;\; \exists n_0 \in \NN: \; \forall n > n_0 \quad (r(A))^n \leqslant \| A^n \| < (r(A) + \varepsilon)^n
    \]
\end{corollary}

\begin{proof}
    По теореме \ref{thm:Beurling-Gelfand_formula}
    \[
        \forall \varepsilon > 0 \;\; \exists n_0 \in \NN: \; \forall n > n_0 \quad \left| \| A^n \|^{\frac{1}{n}} - r(A) \right| < \varepsilon,
    \]
    причём $ \forall n \in \NN \;\; (r(A))^n = r(A^n) \leqslant \| A^n \| $.
    Тогда
    \[
        \forall \varepsilon > 0 \;\; \exists n_0 \in \NN: \; \forall n > n_0 \quad r(A) \leqslant \| A^n \|^{\frac{1}{n}} < r(A) + \varepsilon,
    \]
    откуда получаем искомое неравенство.
\end{proof}

\begin{lemma}
    \label{lem:operator_exponential_norm_convergence}
    Пусть $ A $~--- линейный ограниченный оператор, действующий в банаховом пространстве $ \banachspace $ над $ \CC $.
    Тогда
    \[
        s(A) = \inf \left \{ \omega \in \RR: \lim_{t \to +\infty} e^{-\omega t} \cdot \| \exp(t \cdot A) \| = 0 \right \}
    \]
\end{lemma}

\begin{proof}
    Обозначим $ T(t) = \exp(t \cdot A) $.
    Тогда
    \[
        T(t) = \exp((n \Delta t + \tau) \cdot A) = \left( T(\Delta t) \right)^n \cdot T(\tau),
    \]
    где $ \Delta t > 0 $, $ n = \lfloor t / \Delta t \rfloor $, $ \tau = t - n \Delta t \in [0;\Delta t) $.
    В силу ограниченности $ A $ имеем $ \| T(t) \| \leqslant e^{t \cdot \| A \|} $.
    Согласно теореме \ref{thm:spectral_mapping_theorem}, $ \sigma(T(t)) = \exp(t \cdot \sigma(A)) $,
    из чего следует $ r(T(t)) = e^{t \cdot s(A)} $
    (так как $ \left| e^z \right| = e^{\Re z} $, а экспонента~--- монотонно возрастающая на $ \RR $ функция).
    Применяя следствие \ref{cor:spectral_radius_norm_bounds} и учитывая, что
    \[
        C^{-1} = e^{-\Delta t \cdot \| A \|} \leqslant \| T(-\tau) \|^{-1} = \| T(\tau)^{-1} \|^{-1} \leqslant \| T(\tau) \| \leqslant e^{\Delta t \cdot \| A \|} = C,
    \]
    получаем, что $ \forall \varepsilon > 0 \;\; \exists n_0 \in \NN: \; \forall n > n_0 $
    \[
        C^{-1} \cdot e^{n \Delta t \cdot s(A)} \leqslant \| T(t) \| \leqslant C \cdot e^{n \Delta t \cdot (s(A) + \varepsilon)}
    \]
    Наконец, так как $ n \Delta t = t - \tau $, $ \forall \varepsilon > 0 \;\; \exists t_0 > 0: \; \forall t > t_0 $
    \[
        K^{-1} \cdot e^{t \cdot s(A)} \leqslant \| T(t) \| \leqslant K \cdot e^{t \cdot (s(A) + \varepsilon)},
    \]
    где $ K = C \cdot e^{\Delta t \cdot |s(A)|} = e^{\Delta t \cdot (\|A\| + |s(A)|)} > 0 $. %\limto{\Delta t}{0} 1 $.
    Из этого следует, что $ \forall \varepsilon > 0 $
    \[
        \lim_{t \to +\infty} e^{-t \cdot s(A)} \cdot \| T(t) \| \geqslant K^{-1} > 0,
        \qquad
        \lim_{t \to +\infty} e^{-t \cdot (s(A) + \varepsilon)} \cdot \| T(t) \| \leqslant \lim_{t \to +\infty} K \cdot e^{-t \cdot \varepsilon / 2} = 0
    \]
    Отсюда по определению точной верхней грани получаем доказываемое утверждение.
\end{proof}

Следствие \ref{cor:spectral_radius_norm_bounds} позволяет на основе данных о спектре линейного оператора $ A $ оценить асимптотику $ \| A^n \| $,
а лемма \ref{lem:operator_exponential_norm_convergence}~--- $ \| \exp(t \cdot A) \| $.

Вернёмся к проверочному уравнению Далквиста \eqref{eq:Dahlquist_equation}.
Нас интересуют равномерные оценки на норму численного решения, получаемого заданным методом при заданном постоянном шаге интегрирования.
Для этого введём следующее определение и утверждение:

\begin{definition}
    \label{def:stability_region}
    Множество $ \stabreg = \{ z \in \CC \mid | R(z) | < 1 \} $ называется \emph{областью абсолютной устойчивости} численного метода, обладающего функцией устойчивости $ R(z) $.
    Множество $ \overline{\stabreg} = \{ z \in \CC \mid | R(z) | \leqslant 1 \} $~--- замыкание области абсолютной устойчивости.%
    \footnote{Здесь неявно предполагается непрерывность $ R(z) $.
    Можно обойтись без этого условия, однако тогда некорректно называть $ \overline{\stabreg} $ замыканием $ \stabreg $.}
\end{definition}

\begin{statement}
    \label{stat:linear_numerical_stability}
    Пусть численное решение уравнения \eqref{eq:Dahlquist_equation} ищется интегрированием с постоянным шагом $ \Delta t $
    при помощи метода, обладающего функцией устойчивости $ R(z) $ и соответсвующей областью абсолютной устойчивости $ \stabreg $.
    Пусть также $ R(z) $ регулярна в области $ U \supset \Delta t \cdot \sigma(F_0) $.
    Тогда $ \bvec{x}^n = \left( R(\Delta t \cdot F_0) \right)^n \cdot \bvec{x}_0 $, и
    \begin{align}
        \Delta t \cdot \sigma(F_0) \subseteq \stabreg \qquad \Longleftrightarrow \qquad & \| \left( R(\Delta t \cdot F_0) \right)^n \| \limto{n}{\infty} 0 \\
        \Delta t \cdot \sigma(F_0) \subseteq \CC \setminus \overline{\stabreg} \qquad \Longrightarrow \qquad & \| \left( R(\Delta t \cdot F_0) \right)^n \| \limto{n}{\infty} \infty
    \end{align}
\end{statement}

\begin{proof}
    В силу \eqref{eq:stability_function} имеем первое утверждение: $ \bvec{x}^n = \left( R(\Delta t \cdot F_0) \right)^n \cdot \bvec{x}_0 $.
    Далее заметим, что по теореме \ref{thm:spectral_mapping_theorem}
    \[
        \sigma\left( R(\Delta t \cdot F_0) \right) = R\left( \Delta t \cdot \sigma(F_0) \right)
    \]
    Отсюда следует, что
    \begin{align}
        \Delta t \cdot \sigma(F_0) \subseteq \stabreg \qquad \Longleftrightarrow \qquad & r\left( R(\Delta t \cdot F_0) \right) < 1 \\
        \Delta t \cdot \sigma(F_0) \subseteq \CC \setminus \overline{\stabreg} \qquad \Longleftrightarrow \qquad & r\left( R(\Delta t \cdot F_0) \right) > 1
    \end{align}
    Наконец, применяя следствие \ref{cor:spectral_radius_norm_bounds}, завершаем доказательство утверждения.
\end{proof}

В теории линейной устойчивости важная роль отводится \emph{A-устойчивости}~--- свойству численного решения проверочного уравнения Далквиста не возрастать по норме,
если не возрастает норма истинного решения.
Если к тому же при увеличении шага интегрирования норма численного решения на следующей итерации (или, быть может, через некоторе заранее известное число итераций)
также стремится к нулю, то говорят об \emph{L-устойчивости}.
Дадим формальное определение.

\begin{definition}
    \label{def:A_stability}
    Численный метод называется \emph{A-устойчивым} в случае, если $ \CC^- \equiv \{ z \in \CC \mid \Re z < 0 \} \subseteq \stabreg $.
\end{definition}

\begin{definition}
    \label{def:L_stability}
    Численный метод называется \emph{L-устойчивым} в случае, если он A-устойчив и выполнено $ \displaystyle\lim_{\Re z \to -\infty} R(z) = 0 $.
\end{definition}

\begin{statement}
    \label{stat:A_stability_criterion}
    Пусть $ R(z) $ регулярна в $ \CC^- $.
    Соответствующий численный метод A-устойчив тогда и только тогда, когда $ \forall F_0 $ выполнено
    \[
        \| \exp(t \cdot F_0) \| \limto{t}{+\infty} 0 \qquad \Longrightarrow \qquad \forall \Delta t > 0 \quad \left \| \left( R(\Delta t \cdot F_0) \right)^n \right\| \limto{n}{\infty} 0
    \]
\end{statement}

\begin{proof}
    В силу леммы \ref{lem:operator_exponential_norm_convergence} импликацию из утверждения можно переписать в виде
    \[
        \sigma(F_0) \subseteq \CC^- \quad \Longrightarrow \quad \forall \Delta t > 0 \quad \left \| \left( R(\Delta t \cdot F_0) \right)^n \right\| \limto{n}{\infty} 0
    \]
    Если также воспользоваться \ref{stat:linear_numerical_stability}, получаем
    \[
        \sigma(F_0) \subseteq \CC^- \quad \Longrightarrow \quad \forall \Delta t > 0 \quad \Delta t \cdot \sigma(F_0) \subseteq \stabreg,
    \]
    что при произвольном $ F_0 $ равносильно $ \CC^- \subseteq \stabreg $.
    Это даёт определение \ref{def:A_stability}.
\end{proof}

\begin{statement}
    \label{stat:L_stability_property}
    Пусть $ R(z) $ регулярна в $ \CC^- $.
    Если соответствующий численный метод L-устойчив, то $ \forall F_0 $ выполнено
    \[
        \| \exp(t \cdot F_0) \| \limto{t}{+\infty} 0 \qquad \Longrightarrow \qquad r \left( R(\Delta t \cdot F_0) \right) \limto{\Delta t}{+\infty} 0
    \]
\end{statement}

\begin{proof}
    Аналогично доказательству утверждения \ref{stat:A_stability_criterion} перепишем импликацию в виде
    \[
        \sigma(F_0) \subseteq \CC^- \quad \Longrightarrow \quad r \left( R(\Delta t \cdot F_0) \right) \limto{\Delta t}{+\infty} 0
    \]
    В силу теоремы \ref{thm:spectral_mapping_theorem} это эквивалентно
    \[
        \sigma(F_0) \subseteq \CC^- \quad \Longrightarrow \quad \sup \{ |\lambda| \mid \lambda \in R(\Delta t \cdot \sigma(F_0)) \} \limto{\Delta t}{+\infty} 0,
    \]
    что при произвольном $ F_0 $ равносильно $ \forall z \in \CC^- \; | R(\Delta t \cdot z) | \limto{\Delta t}{+\infty} 0 $.
    Это верно для L-устойчивых методов по определению \ref{def:L_stability}.
\end{proof}

Стоит отметить, что утверждение \ref{stat:L_stability_property}, в отличие от \ref{stat:A_stability_criterion}, сформулировано в форме признака, а не критерия.
Также в нём получено лишь утверждение о пределе спектрального радиуса, а не нормы.
На деле это означает, что получаемая для L-устойчивого метода матрица перехода с увеличением размера шага становится <<почти нильпотентной>>.
Если $ F_0 $ диагонализуемая, то из стремления к нулю спектрального радиуса матрицы перехода будет автоматически следовать и стремление к нулю её нормы.


\subsection{Логарифмическая норма}
\label{subsec:logarithmic_norm}

Выше были приведены основные результаты линейной теории устойчивости.
Все они в значительной степени опираются на спектральный анализ.
С одной стороны, это облегчает исследование свойств численных методов,
так как теорема об отображении спектра позволяет по функции устойчивости метода
определить асимптотические свойства матрицы перехода.
С другой стороны, спектральный анализ не позволяет получить оценки,
справедливые в течение всего рассматриваемого времени,
включая начальный этап эволюции линейной системы.

Для получения более строгих в указанном смысле оценок вводится \emph{логарифмическая норма}:

\begin{definition}
    \label{def:logarithmic_norm}
    Пусть $ A $~--- линейный ограниченный оператор, действующий в банаховом пространстве $ \banachspace $ над $ \CC $.
    Число
    \begin{equation}
        \label{eq:logarithmic_norm}
        \lognorm{A} = \lim_{h \to +0} \frac{\| I + h \cdot A \| - 1}{h}
    \end{equation}
    называется \emph{логарифмической нормой} оператора $ A $.
\end{definition}

Далее приведём без доказательства несколько утверждений,
связывающих логарифмическую норму с теорией линейной устойчивости.
Все утверждения взяты из работ \cite{soderlind2006lognorm, lambert1991methods}.

\begin{statement}
    \label{stat:logarithmic_norm_properties}
    Пусть $ A $, $ B $~--- линейные операторы, действующие в конечномерном банаховом пространстве $ \banachspace $ над $ \CC $.
    Тогда верно следующее:
    \begin{enumerate}[itemsep=0em]
        \item $ \lognorm{A} $ определена.
        \item $ \lognorm{A} \leqslant \| A \| $.
        \item $ \lognorm{\gamma \cdot A} = \gamma \cdot \lognorm{A} $ для любого $ \gamma > 0 $.
        \item $ \lognorm{A + z \cdot I} = \lognorm{A} + \Re z $.
        \item $ \lognorm{A + B} \leqslant \lognorm{A} + \lognorm{B} $.
        \item $ s(A) \leqslant \lognorm{A} $. \label{itm:logarithmic_norm_and_spectral_bound}
        \item $ \| \exp(t \cdot A) \| \leqslant e^{t \cdot \lognorm{A}} $ для любого $ t \geqslant 0 $. \label{itm:logarithmic_norm_exponential_bound}
    \end{enumerate}
\end{statement}

Пункт \ref{itm:logarithmic_norm_and_spectral_bound} связывает логарифмическую норму со спектральными свойствами $ A $.
Пункт \ref{itm:logarithmic_norm_exponential_bound} даёт справедливую всё неотрицательное время оценку на оператор эволюции,
получаемый из $ A $, но, в силу пункта \ref{itm:logarithmic_norm_and_spectral_bound},
а также леммы \ref{lem:operator_exponential_norm_convergence},
данная оценка может не являться асимптотически оптимальной.


\subsection{Линейная жёсткость}
\label{subsec:linear_stiffness}

Как видно из приведённых результатов, спектр матрицы $ F_0 $ может задавать определённые ограничения на шаг интегрирования.
Действительно, если $ \lognorm{F_0} < 0 $, но численный метод не A-устойчив, полученное с его помощью решение может вести себя некорректно при некоторых $ F_0 $ и $ \Delta t $:
если $ \Delta t \cdot \sigma(F_0) \not\subseteq \stabreg $, то численное решение может возрастать,
в то время как норма точного решения ограничена сверху убывающей экспонентой.
С другой стороны, такая ситуация невозможна независимо от $ \Delta t $ при использовании A-устойчивых методов.
Но A-устойчивость не гарантирует соизмеримую с точным решением скорость затухания численного; возможен даже случай $ \displaystyle \lim_{\Re z \to -\infty} |R(z)| = 1 $,
что приводит к слабо затухающим осцилляциям численного решения вокруг нуля при сравнительно быстром стремлении к нулю истинного решения.
Если требуется рост скорости затухания за конечное число шагов при увеличении $ \Delta t $, следует пользоваться L-устойчивыми методами.

На рисунках \ref{fig:linear_instability_example}, \ref{fig:linear_instability_example_2} проиллюстрировано поведение явного метода Эйлера (не A-ус\-той\-чи\-вый),
метода трапеций (A-устойчивый, но не L-устойчивый) и неявного метода Эйлера (L-устойчивый) при разных значениях $ \Delta t \cdot F_0 $ в одномерной задаче Далквиста.

\begin{figure}[ht!]
    \centering
    \begin{gnuplot}[terminal=epslatex, terminaloptions={color dashed size 16cm,6cm}]
        load './gnuplot/common.gp'

        set style increment default
        set style data lines
        set xlabel  '$ t $'
        set xrange  [ 0 : 10 ] noreverse writeback
        set ylabel  '$ x(t) $' #rotate by 0
        set yrange  [ * : * ] noreverse writeback

        set key width -16

        # Параметры.
        z = -1.5
        N = 5                    # Число точек.
        T = 9.0                  # Время интегрирования.
        lamb = z * (N - 1) / T   # Показатель экспоненты.

        load './gnuplot/Dahlquist.gp'

        set xtics 1
        set xzeroaxis lw 3

        plot Trapezoid using (times[$1]):(Trapezoid[$1]) with linespoints t 'метод трапеций' lw 3 ps 2, \
             BackwardEuler using (times[$1]):(BackwardEuler[$1]) with linespoints t 'неявный метод Эйлера' lw 3 ps 2, \
             ForwardEuler using (times[$1]):(ForwardEuler[$1]) with linespoints t 'явный метод Эйлера' lw 3 ps 2, \
             f(x) t 'точное решение' lw 4 lc 'black'
    \end{gnuplot}

    \caption{Поведение простейших численных методов при решении одномерного уравнения Далквиста ($ \Delta t \cdot F_0 = -1.5 $).}
    \label{fig:linear_instability_example}
\end{figure}

\begin{figure}[ht!]
    \centering
    \begin{gnuplot}[terminal=epslatex, terminaloptions={color dashed size 16cm,6cm}]
        load './gnuplot/common.gp'

        set style increment default
        set style data lines
        set xlabel  '$ t $'
        set xrange  [ 0 : 10 ] noreverse writeback
        set ylabel  '$ x(t) $' #rotate by 0
        set yrange  [ * : * ] noreverse writeback

        set key width -16

        # Параметры.
        z = -15.0
        N = 5                    # Число точек.
        T = 9.0                  # Время интегрирования.
        lamb = z * (N - 1) / T   # Показатель экспоненты.

        load './gnuplot/Dahlquist.gp'

        set xtics 1
        set xzeroaxis lw 3

        plot Trapezoid using (times[$1]):(Trapezoid[$1]) with linespoints t 'метод трапеций' lw 3 ps 2, \
             BackwardEuler using (times[$1]):(BackwardEuler[$1]) with linespoints t 'неявный метод Эйлера' lw 3 ps 2, \
             f(x) t 'точное решение' lw 4 lc 'black'
    \end{gnuplot}

    \caption{Поведение простейших численных методов при решении одномерного уравнения Далквиста ($ \Delta t \cdot F_0 = -15 $).}
    \label{fig:linear_instability_example_2}
\end{figure}

Зачастую область устойчивости не A-устойчивых методов ограничена (в частности, это верно для всех явных линейных численных методов) или содержит лишь некоторый подсектор $ \CC^- $,
поэтому ограничение на шаг интегрирования оказывается ограничением сверху.
Таким образом, спектральные свойства $ F_0 $ обуславливают максимально допустимый шаг численного интегрирования.
Ситуацию осложняет следующая теорема,
требующая в случае одностадийных схем делать выбор между устойчивостью, высоким порядком аппроксимации и линейностью схемы:

\begin{theorem}[второй барьер Далквиста]
    \label{thm:Dahlquist_second_barrier}
    Не существует A-устойчивых линейных многошаговых одностадийных схем с порядком аппроксимации выше второго.
\end{theorem}

Ограничение на $ \Delta t $ может сохранятся даже при решении нелинейных задач вида \eqref{eq:autonomous_initial_value_problem}.
Исходя из вышеизложенного анализа можно ввести следующие определения характерных масштабов времени:

\begin{definition}
    \label{def:linear_and_nonlinear_timescales}
    Обозначим характерное время изменения $ F = \frac{\partial f}{\partial \bvec{x}}(\bvec{x}(t)) $ как~$ \taunonlin $,
    %а $ r \! \left( \frac{\partial f}{\partial \bvec{x}} \right) = 1 / r(F) $~---
    а $ 1 / r(F) $~---
    характерное время реакции линеаризованной системы на небольшие возмущения~--- как~$ \taulin $.
\end{definition}

%Рассмотрим случай, когда $ \taunonlin $~--- характерное время изменения $ F = \frac{\partial f}{\partial \bvec{x}}(\bvec{x}(t)) $~---
%много больше $ \taulin = 1/r(F) $~--- характерного времени реакции системы на небольшие возмущения.
Рассмотрим случай, когда $ \taulin \ll \taunonlin $.
Тогда линеаризация \eqref{eq:linearized_initial_value_problem} остаётся достаточно точной гораздо дольше $ \taulin $.
Это автоматически оставляет в силе ограничения на шаг интегрирования, полученные для линейных систем.
В частности, если $ \lognorm{F} < 0 $, но $ \Delta t \cdot \sigma(F) \not\subseteq \stabreg $, численное решение может вести себя неустойчиво к небольшим возмущениям,
в то время как точное решение, наоборот, будет обладать эффектом демпфирования.

Приведённые выше рассуждения показывают, что определённые системы дифференциальных уравнений могут обладать свойствами,
вынуждающими использовать малый шаг интегрирования при их решении недостаточно устойчивыми в смысле \ref{def:A_stability} и \ref{def:L_stability} методами.
Традиционно такие системы называют \emph{жёсткими}.
Как указано в \cite{heirer1999solvingode2, lambert1991methods}, существует несколько определений жёсткости,
каждое из которых обладает своими достоинствами и недостатками.
Приведём здесь одно из них:

\begin{definition}
    \label{def:stiffness}
    Система вида $ \frac{d \bvec{x}}{d t} = f(t, \bvec{x}) $ называется \emph{жёсткой} в том случае,
    если для получения корректного (то есть в заданной степени близкого к точному решению как качественно, так и количественно)
    численного решения необходимо использовать шаг интегрирования,
    много меньший характерных масштабов времени, на которых меняется точное решение.
\end{definition}

Данное определение слишком общее и не отвечает на вопросы о природе ограничения на шаг интегрирования.
На основе всего вышеизложенного анализа мы дадим более узкое, но в некоторой степени и более информативное определение жёсткости.

\begin{definition}
    \label{def:linear_stiffness}
    Система вида $ \frac{d \bvec{x}}{d t} = f(\bvec{x}) $ называется \emph{линейно жёсткой} в том случае,
    если характерное время реакции линеаризованной системы на небольшие возмущения $ \taulin $
    много меньше характерных масштабов времени,
    %на которых меняется точное решение.
    на которых ищется решение.
\end{definition}

Зачастую численное решение ищется на интервалах,
сопоставимых с характерными временн\'{ы}ми масштабами наиболее медленно протекающих процессов, описываемых системой
(это действительно так, если в численном решении требуется полноценно отобразить всю динамику системы).
Также для численных методов с ограниченной областью устойчивости условие $ \Delta t \cdot \sigma(F) \subseteq \stabreg $ влечет $ \Delta t \sim \taulin $.
Тогда \ref{def:linear_stiffness} оказывается частным случаем \ref{def:stiffness},
причём необходимость выбора малого шага оказывается обусловленной <<жёстким>> линейным поведением системы в окрестности истинных решений.
В таком случае некорректность численного решения понимается в смысле неустойчивости к малым возмущениям там,
где точное решение к ним устойчиво.%
\footnote{Эта интерпретация будет позже формализована при помощи введения понятия \emph{сжимаемости}
(см. секцию \ref{subsec:nonlinear_stability}, определения \ref{def:contractivity} и \ref{def:contractivity_sequence}).}

Стоит также отметить, что в случае линейных или слабо нелинейных
(то есть линеаризацию которых можно долго считать достаточно точной) систем
характерные масштабы времени наиболее медленно протекающих процессов задаются наименьшими по модулю элементами спектра матрицы $ F $.
Тогда для линейно жёстких систем спектр $ F $ распадается на две части: \emph{ведущую} и \emph{паразитическую}.

\begin{definition}
    \label{def:driving_and_parasitic_spectrum}
    Будем называть спектр оператора $ A $ \emph{распадающимся на ведущую и паразитическую части}, если
    существует расщепление $ \sigma(A) = \sigma_d(A) \sqcup \sigma_p(A) $ такое, что
    \[
        r_d(F) \equiv \sup \{ | \lambda | \mid \lambda \in \sigma_d(F) \} \ll \inf \{ | \lambda | \mid \lambda \in \sigma_p(F) \} \equiv b_p(F)
    \]
    %Число $ b_p / r_d \gg 1 $ при этом отражает отношение масштабов времени быстро и медленно текущих процессов.
\end{definition}

%где $ r_d(F) = \sup \{ | \lambda | \mid \lambda \in \sigma_d(F) \} \ll \sup \{ | \lambda | \mid \lambda \in \sigma_p(F) \} = r_p(F) $.
Более того, в случае седловых задач доля паразитического спектра может иметь положительную действительную часть.
Это может вызывать <<взрывное>> поведение даже A-устойчивых методов,
ведь в таком случае попадание $ \Delta t \cdot \sigma(F) $ в область устойчивости не гарантируется.
В задачах, где известно, что такое поведение точного решение нефизично и отсутствует в виду выбора правильных начальных условий,
данное свойство численного решения является нежелательным.


\subsection{Экспоненциальные интеграторы}
\label{subsec:exponential_integrators}

Линейный анализ устойчивости известен давно, а потому достаточно подробно разработан
\cite{auzinger1993modern, dahlquist1963special, dahlquist1975stability, liu2019study, heirer1999solvingode2, lambert1991methods}.
Как следствие, велик и арсенал тех методов, которые способны в той или иной степени решить проблемы, вызываемые линейной жёсткостью.
В основном это различные многошаговые или многостадийные линейные схемы, обладающие теми или иными свойствами линейной устойчивости:
неявные методы Рунге-Кутты, методы Гира, формулы дифференцирования назад и прочее.
%Все они обладают достаточно большой областью устойчивости, что, как видно из приведённого ранее анализа,
%делает их полезными для интегрирования линейно жёстких систем.
Обладая достаточной устойчивостью, они одновременно могут иметь высокий порядок аппроксимации.
Тем не менее, их область устойчивости меньше области устойчивости неявного метода Эйлера,
что мотивирует разработку новых методов.

В настоящей работе предлагается отойти от стандартных линейных схем с постоянными коэффициентами и обратить внимание на численные методы,
утилизирующие информацию о матрице Якоби правой части системы \eqref{eq:autonomous_initial_value_problem}.
Она может предоставить достаточный объём информации о локальном линейном поведении системы.
Эта информация полезна в том числе и для решения проблемы линейной жёсткости.
Стоит добавить, что, при решении жёстких систем обычно используются неявные методы.
Возникающие при этом алгебраические уравнения зачастую решаются методом Ньютона.
Это означает, что матрица Якоби правой части вычисляется в любом случае,
поэтому её использование для модификации численной схемы не приносит дополнительной вычислительной сложности
(кроме, быть может, последующих операций с матрицей).

Методы, использующие матрицу Якоби правой части интегрируемой системы называются \emph{адаптивными}.
Среди них можно перечислить методы Розенброка, Обрешкова, а также \emph{экспоненциальные интеграторы}.
В данной работе основное внимание будет уделено именно последним.

\begin{definition}
    \label{def:exponential_integrator}
    Экспоненциальным интегратором называется численный метод,
    дающий применительно к линейной задаче \eqref{eq:linearized_initial_value_problem} точное решение.
\end{definition}

\begin{remark}
    \label{rem:exponential_integrator_stability_function}
    Любой экспоненциальный интегратор обладает функцией устойчивости $ R(z) = e^z $.
    Он также является L-устойчивым.
\end{remark}

Из определения и замечания видно, что экспоненциальный интегратор точно интегрирует линейную часть системы.
Это позволяет исключить влияние линейной жёcткости на решение.
Преимуществом экспоненциальных интеграторов также является полное отсутствие численной
(то есть обусловленной исключительно методом, а не задачей) диссипации в линейных системах.
В некоторых задачах, близких к линейным, это качественно отличает данные методы от
стандартных A-устойчивых схем с большими областями устойчивости
(пример можно увидеть в разделе \ref{sec:Lotka-Volterra}).

Некоторые авторы также расширяют определение экспоненциальных интеграторов на все методы,
использующие экспоненту матрицы Якоби правой части системы.
Обзор существующих экспоненциальных интеграторов можно найти, например, в работах \cite{minchev2005expint, hochbruck_ostermann_2010}.
Мы же остановимся только на простейших из них, необходимых для дальнейшего анализа.

%Рассмотрим две схемы, полученных из явного и неявного методов Эйлера соответственно
%(так называемые <<exponential time differencing>>-методы Рунге-Кутты первого порядка):
Рассмотрим явный и неявный экспоненциальные методы Эйлера:
%
\begin{align}
    \text{\emph{Явный:}}   && \bvec{x}^{n+1} - \bvec{x}^n &= \left( \exp(\Delta t \cdot F) - I \right) F^{-1} \cdot f(\bvec{x}^n) \label{eq:ETDRK1_explicit} \\
    \text{\emph{Неявный:}} && \bvec{x}^{n+1} - \bvec{x}^n &= \left( I - \exp(- \Delta t \cdot F) \right) F^{-1} \cdot f(\bvec{x}^{n+1}) \label{eq:ETDRK1_implicit}
\end{align}
%
Здесь, как и ранее, $ F = \frac{\partial f}{\partial \bvec{x}} $~--- матрица Якоби правой части системы.%
%\footnote{Отметим, что мы явно не указываем, в какой именно точке вычисляется $ F $.
%    Для линейных систем это не важно, но для нелинейных является определённой степенью свободы,
%    заложенной в методе.}
При этом за пользователем остаётся свобода выбора точки, в которой вычисляются производные:
для линейных систем это не важно, но для нелинейных является определённой степенью свободы, заложенной в методе.
На самом деле, возможно даже использование приближённого значения матрицы,
но при этом экспоненциальный интегратор не будет давать точного решения на тестовой задаче Далквиста.

Зная \eqref{eq:linearized_solution}, несложно проверить, что при рассмотрении линеаризованной задачи
$ f(\bvec{x}) = f_0 + F_0 \cdot (\bvec{x} - \bvec{x}_0) $
оба метода дают точное решение.
Следует отметить, что коэффициент при $ f $ в обоих методах может расти экспоненциально быстро с увеличением $ \Delta t $.
В таком случае с увеличением шага численное решение может быстро выйти за пределы окрестности, в которой применима линеаризация правой части.
Это следует отнести к недостаткам данных методов, так как такое поведение потенциально может являться источником неустойчивости в случае сильной нелинейности правой части.
%Пример такому поведению будет дан в следующих разделах.
В качестве одного из результатов работы также будет приведён двухточечный экспоненциальный интегратор,
построенный по аналогии с \eqref{eq:ETDRK1_explicit} и \eqref{eq:ETDRK1_implicit}, но лишённый упомянутого недостатка.



\section{Нелинейная теория устойчивости}
\label{sec:nonlinear_stability}

В предыдущем разделе были приведены основные положения линейной теории устойчивости,
введено понятие жёсткости и линейной жёсткости,
упомянуты адаптивные численные методы, использующие знание матрицы Якоби правой части системы для борьбы с линейной жёсткостью.
Вместе с этим, в общих чертах обозначены предполагаемые границы применимости линейной теории устойчивости:
\begin{enumerate}
    \item
        Линейный анализ устойчивости некорректно применять при наличии возмущений,
        величина или характер которых ставит под сомнение точность линеаризации системы.
    \item
        Линейный анализ устойчивости некорректно применять,
        если характерное время линейной реакции на малые возмущения сопоставимо с или больше характерных масштабов времени,
        на которых меняется матрица Якоби правой части системы.
\end{enumerate}

В главе 7 работы \cite{lambert1991methods} приведены примеры систем,
применение линейного анализа к которым даёт некорректные результаты
(по крайней мере, если ограничиваться лишь спектральными признаками и не использовать оценки, приведённые в разделе \ref{subsec:logarithmic_norm}).
В этой связи в настоящем разделе рассмотрены основные результаты нелинейной теории устойчивости.
Вместе с тем также приведены некоторые сведения,
касающиеся поведения неявных численных методов при интегрировании существенно нелинейных систем.


\subsection{Нелинейная устойчивость}
\label{subsec:nonlinear_stability}

В линейном анализе устойчивости существенная роль отводится линейным системам,
решения которых стремятся к нулю при устремлении времени к плюс бесконечности.
Такое поведение также означает, что любые два решения с течением времени <<сближаются>> друг с другом.
В секции \ref{subsec:linear_stiffness} этому важному феномену дана интерпретация:
при небольших возмущениях начальных условий решение будет стремится к невозмущенному с течением времени.
При использовании устойчивых в различных смыслах методов это в той или иной степени гарантирует аналогичное поведение и у численного решения.
Поэтому при появлении сомнений в корректности линейного анализа устойчивости логично попытаться обобщить определение данного явления на произвольную систему.

\begin{definition}
    \label{def:contractivity}
    Пусть $ \bvec{x}(t) $ и $ \bvec{y}(t) $~--- решения системы $ \frac{d \bvec{x}}{d t} = f(t, \bvec{x}) $.
    Данные решения называются \emph{сжимающимися} на отрезке $ [a; b] $ в случае
    \[
        \forall t_1, t_2: \; a \leqslant t_1 \leqslant t_2 \leqslant b \qquad \| \bvec{x}(t_2) - \bvec{y}(t_2) \| \leqslant \| \bvec{x}(t_1) - \bvec{y}(t_1) \|
    \]
\end{definition}

Аналогичное определение можно ввести и для численных решений (и, вообще говоря, для любых последовательностей):

\begin{definition}
    \label{def:contractivity_sequence}
    Пусть $ \{\bvec{x}_n\}_{n \in \NN_0} $ и $ \{\bvec{y}_n\}_{n \in \NN_0} $~--- последовательности элементов некоторого линейного нормированного пространства.
    Данные последовательности называются \emph{сжимающимися} на отрезке $ [a; b] $ в случае
    \[
        \forall n_1, n_2: \; a \leqslant n_1 \leqslant n_2 \leqslant b \quad \| \bvec{x}_{n_2} - \bvec{y}_{n_2} \| \leqslant \| \bvec{x}_{n_1} - \bvec{y}_{n_1} \|
    \]
\end{definition}

Для систем вида $ \frac{d \bvec{x}}{d t} = f(t, \bvec{x}) $ известно \cite{auzinger1990note, auzinger1993modern} достаточное условие
сжимаемости точных решений.

\begin{definition}
    \label{def:one-sided_lipschitz}
    Пусть $ \hilbertspace $~--- гильбертово пространство над $ \CC $.
    Функция $ f(t, \bvec{x}) = f: \RR \times D \to \hilbertspace $
    (где $ D \subseteq \hilbertspace $~--- выпуклая область)
    и система $ \frac{d \bvec{x}}{d t} = f(t, \bvec{x}) $ называются \emph{односторонне липшициевыми} на отрезке $ [a; b] $ в случае
    \[
        \exists \nu(t): \; \forall \bvec{x}, \bvec{y} \in D, \; \forall t \in [a, b] \quad
        \Re \dotprod{f(t, \bvec{x}) - f(t, \bvec{y})}{\bvec{x} - \bvec{y}} \leqslant \nu(t) \| \bvec{x} - \bvec{y} \|^2
    \]
\end{definition}

\begin{definition}
    \label{def:dissipative}
    Пусть в определении \ref{def:one-sided_lipschitz} $ \nu(t) \leqslant 0 $.
    Тогда функция $ f(t, \bvec{x}) $ и система $ \frac{d \bvec{x}}{d t} = f(t, \bvec{x}) $ называются \emph{диссипативными}.
\end{definition}

\begin{statement}[достаточный признак сжимаемости]
    \label{def:contractivity_condition}
    Пусть система $ \frac{d \bvec{x}}{d t} = f(t, \bvec{x}) $ диссипативна на $ [a; b] $.
    Тогда все её решения являются сжимающимися на $ [a; b] $.
\end{statement}

\begin{remark}
    \label{rem:linear_one-sided_lipschitz}
    Линейная функция $ f(t, \bvec{x}) = A(t) \cdot \bvec{x} + f_0(t) $ односторонне липшициева на всей области определения $ A(t) $
    с наименьшим возможным коэффициентом $ \nu(t) = \lognorm{A(t)} $.
\end{remark}

Замечание \ref{rem:linear_one-sided_lipschitz} позволяет связать линейную теорию устойчивости с нелинейной.
Его естественным продолжением является следующая теорема:

\begin{theorem}[Далквист, 1959]
    \label{thm:nonlinear_to_linear}
    Пусть $ \hilbertspace $~--- гильбертово пространство над $ \CC $.
    Пусть $ f(t, \bvec{x}) = f: \RR \times D \to \hilbertspace $,
    где $ D \subseteq \hilbertspace $~--- выпуклая область.
    Пусть также существуют $ a, b \in \RR $ и $ \nu(t) $~--- такая кусочно-непрерывная функция, что
    \[
        \forall t \in [a; b], \; \forall \bvec{x} \in D \quad \lognorm{\frac{\partial f}{\partial \bvec{x}}(t, \bvec{x})} \leqslant \nu(t)
    \]
    Тогда, если $ \bvec{x}(t) $ и $ \bvec{y}(t) $~--- два решения системы $ \frac{d \bvec{x}}{d t} = f(t, \bvec{x}) $,
    то для любых $ t_1, t_2 \in \RR: \; a \leqslant t_1 \leqslant t_2 \leqslant b $ выполнено
    \[
        \| \bvec{x}(t_2) - \bvec{y}(t_2) \| \leqslant \exp\left( \int\limits_{t_1}^{t_2} \nu(s) ds \right) \| \bvec{x}(t_1) - \bvec{y}(t_1) \|
    \]
\end{theorem}

Наконец, приведём достаточное условие сжимаемости численных решений.

\begin{theorem}[Далквист, 1978]
    Пусть рассматривается конечномерное гильбертово пространство над $ \RR $.
    Если численный метод A-устойчивый, то существует норма (называемая \emph{G-нормой}),
    в которой любые два численных решения диссипативной на $ [a; b] $ системы будут сжимающимися на $ [a; b] $.
\end{theorem}

Сразу отметим, что известны и другие признаки сжимаемости численных решений,
среди которых есть и не требующие введение другой нормы.
Однако для общего анализа их включение в данную работу не является необходимым.
%В силу специфики численных методов, которые будут предложены в данной работе,
%упомянутые признаки не являются релевантными.

Приведённые теоремы позволяет обосновать линейный анализ устойчивости даже для нелинейных систем.
Стоит, однако, отметить, что это возможно исключительно благодаря ослаблению утверждений до формы достаточных условий,
а также благодаря использованию логарифмической нормы вместо спектральной границы.
Напомним, что первая даёт справедливую в течение всего времени оценку на норму оператора эволюции линейной системы,
а вторая~--- только асимптотическую оценку (пусть и не менее строгую).


\subsection{Нелинейная жёсткость}
\label{subsec:nonlinear_stiffness}

В предыдущей секции были приведены важные утверждения,
позволяющие формально распространить определение линейной жёсткости на нелинейные системы,
а также показывающие, что A-устойчивые методы позволяют в известной степени устранять эффекты линейной жёсткости даже для нелинейных систем.

Практика, однако, показывает, что при решении систем с сильно нелинейной в том или ином смысле правой частью
могут возникать различные нежелательные эффекты неустойчивости,
даже если используются A-устойчивые, L-устойчивые или адаптивные методы.
Данные эффекты также вынуждают ограничивать величину шага.
Таким образом, система проявляет свойства жёсткости, которые невозможно объяснить одной только линейной составляющей.

Для примера рассмотрим задачу Коши
%
\begin{equation}
    \label{eq:cosine_system}
    \begin{dcases}
        \frac{d x}{d t} = \cos\left( \frac{\pi}{2} \cdot x \right) \\
        x(0) = x_0 = 0
    \end{dcases}
    \qquad
    F(x) = \frac{\partial f}{\partial x} = - \frac{\pi}{2} \sin\left( \frac{\pi}{2} \cdot x \right),
\end{equation}
%
имеющую точное решение
\begin{equation}
    \label{eq:cosine_system_solution}
    x(t) = \frac{2}{\pi} \arcsin\left( \tanh\left( \frac{\pi}{2} t \right) \right)
\end{equation}

Рассматриваемая система является автономной и имеет множество положений равновесия $ x = 2k + 1, \; k \in \ZZ $.
Из них устойчивые только $ x = 4k + 1, \; k \in \ZZ $.
В окрестности каждого из положений равновесия система достаточно хорошо линеаризуема (с кубической точностью).
Более того, в любой точке функция $ f(x) $ отличается от свой линеаризации в ближайшем положении равновесия не более, чем на
$ |\cos(\pi/2) - \pi/2| \approx 1.58 \approx 1.58 \cdot \max |f(x)| $.
%Наконец, на участке $ [0; 1) $ (от начальной точки и до одного из устойчивых положений равновесия)
%$ f(x) > 0 $ и сублинейно стремится к нулю при $ x \to 1 $ (так как $ -\pi /2 \leqslant F(x) < 0 $ при $ x \in [0;1) $).
%Значит, точное решение не покинет отрезка $ [0; 1] $ и будет с экспоненциальной скоростью стремиться к $ x = 1 $.
Наконец, в любой момент времени точное решение находится в промежутке $ [0; 1) $,
с экспоненциальной скоростью стремясь к положению равновесия $ x = 1 $.
Для $ x \in [0; 1) $ имеем $ -\pi /2 < F(x) \leqslant 0 $, то есть $ F(x) \in \CC^- $.
Исходя из этого можно выдвинуть предположение, что численное решение данной системы при помощи A-устойчивого метода не вызвовет никаких проблем,
даже если взять шаг интегрирования, сравнимый с $ \min (1 / |F|) = 2 / \pi \sim \taulin $.

Воспользуемся L-устойчивым неявным методом Эйлера.
Шаг интегрирования возьмём $ \Delta t = 2 $,
возникающие при интегрировании нелинейные уравнения будем решать методом Ньютона с начальным приближением в текущей точке.
Для сравнения построим также график точного решения.
%референсное решение методом трапеций с шагом $ \Delta t = 0.02 $
Оба графика можно видеть на рисунке \ref{fig:nonlinear_instability_example}.
Несложно заметить, что полученное методом Эйлера решение некорректно.
Использование экспоненциальных интеграторов из раздела \ref{subsec:exponential_integrators}
также не приносит положительных результатов (см. рис. \ref{fig:nonlinear_instability_example_2}),
несмотря на то, что данные методы точно интегрируют линейную составляющую системы
(для сравнения также добавлен график численного решения, полученного методом, описанным в следующей главе).

\begin{figure}[ht!]
    \centering
    \begin{gnuplot}[terminal=epslatex, terminaloptions={color dashed size 16cm,6cm}]
        load './gnuplot/common.gp'

        #set style increment default
        set style data linespoints
        set xlabel  '$ t $'
        set xrange  [ 0 : * ] noreverse writeback
        set ylabel  '$ x(t) $' #rotate by 0
        set yrange  [ -2 : 2 ] noreverse writeback

        set key width -16

        f(x) = 2 * asin( tanh(pi * x / 2.0) ) / pi

        set xtics 2
        set ytics 1
        set xzeroaxis lw 3

        plot './data/exponential_integrators/tests.csv' every ::1 using 1:3 t 'неявный метод Эйлера' lw 3 ps 2, \
             f(x) t 'точное решение' with lines lw 4 lc 'black'
    \end{gnuplot}

    \caption{Поведение L-устойчивого метода Эйлера при решении уравнения \eqref{eq:cosine_system}}
    \label{fig:nonlinear_instability_example}
\end{figure}

\begin{figure}[ht!]
    \centering
    \begin{gnuplot}[terminal=epslatex, terminaloptions={color dashed size 16cm,9.0cm}]
        load './gnuplot/common.gp'

        #set style increment default
        set style data linespoints
        set xlabel  '$ t $'
        set xrange  [ 0 : * ] noreverse writeback
        set ylabel  '$ x(t) $' #rotate by 0
        set yrange  [ -2 : 4 ] noreverse writeback

        set key width -32

        f(x) = 2 * asin( tanh(pi * x / 2.0) ) / pi

        set xtics 2
        set ytics 1
        set xzeroaxis lw 3

        plot './data/exponential_integrators/tests.csv' every ::1 using 1:6 t 'неявный экспоненциальный метод Эйлера' lw 3 ps 2, \
             './data/exponential_integrators/tests.csv' every ::1 using 1:5 t 'явный экспоненциальный метод Эйлера' lw 3 ps 2, \
             './data/exponential_integrators/tests.csv' every ::1 using 1:7 t 'предложенный в разделе \ref{sec:weighted_Euler} метод' lw 3 lc 'web-blue' ps 2, \
             f(x) t 'точное решение' with lines lw 4 lc 'black'
        #'./data/exponential_integrators/tests.csv' every ::1 using 1:8 t 'предложенный в разделе \ref{sec:modified_Newton} метод' lw 3 lc 'dark-plum'ps 2, \
    \end{gnuplot}

    \caption{Поведение простейших экспоненциальных интеграторов при решении уравнения \eqref{eq:cosine_system}}
    \label{fig:nonlinear_instability_example_2}
\end{figure}

Данный пример показывает, что явление жёсткости не ограничивается только линейной её составляющей.
Выпишем невязку неявного метода Эйлера и его матрицу Якоби:
%
\begin{equation}
    \label{eq:implicit_Euler_residual}
    \res(t^{n+1}, \bvec{x}^{n+1}) = \bvec{x}^{n+1} - \bvec{x}^n - \Delta t \cdot f(t^{n+1}, \bvec{x}^{n+1})
\end{equation}
%
\begin{equation}
    \label{eq:implicit_Euler_jacobian}
    \jac(t^{n+1}, \bvec{x}^{n+1}) = \frac{\partial \res}{\partial \bvec{x}}(t^{n+1}, \bvec{x}^{n+1}) = I - \Delta t \cdot F(t^{n+1}, \bvec{x}^{n+1})
\end{equation}
%
В нашем случае $ f(x) $ и $ F(x) $ имеют целое семейство корней.
Поэтому при достаточно большом $ \Delta t $ несколько корней также имеет $ \res(x) $,
а $ \jac(x) $ перестаёт быть отделимым от нуля.
Теорема Канторовича \cite{kantorovich1949method,ortega2000iterative} даёт достаточные условия сходимости метода Ньютона к корню уравнения.
Одним из условий как раз является отделимость матрицы Якоби от нуля.
Таким образом, из-за нелинейности правой части и большого шага интегрирования, с одной стороны,
теряются достаточные условия сходимости метода Ньютона, и, с другой стороны, появляются некорректные корни невязки.
Приведённый пример показывает, что это может оказаться достаточным для того, чтобы метод выдавал некорректные решения.
Как показано в \cite{lambert1991methods}, более устойчивый метод простой итерации для поиска корней
также может расходиться в случае жёстких систем и большого шага интегрирования.
Можно предположить, что это универсальное свойство некоторых систем,
заставляющее делать выбор между величиной шага и сложностью алгоритма поиска корней нелинейной невязки.
%Можно предположить, что эта участь преследует все локальные
%(то есть не имеющие для широкого класса уравнений гарантированной возможности найти все корни)
%методы.

Таким образом, мы приходим к новому определению жёсткости системы,
связанному теперь с её нелинейным характером.

\begin{definition}
    \label{def:nonlinear_stiffness}
    Система вида $ \frac{d \bvec{x}}{d t} = f(t, \bvec{x}) $ называется \emph{нелинейно жёсткой} в том случае,
    если нелинейные свойства правой части существенно влияют на
    %допустимую величину шага интегрирования,
    %при которой возможна сходимость используемого метода поиска корней невязки выбранной численной схемы,
    %либо на сложность используемого метода поиска корней,
    %способного для выбранного шага интегрирования корректно решать возникающие нелинейные уравнения.
    % \begin{itemize}
    %     \item
    %         допустимую величину шага интегрирования,
    %         при которой возможна сходимость используемого алгоритма поиска корней невязки выбранной численной схемы;
    %     \item
    %         допустимую степень неявности используемой численной схемы,
    %         при которой для заданного шага интегрирования возможна сходимость используемого алгоритма поиска корней невязки.
    %     \item
    %         сложность алгоритма поиска корней,
    %         достаточного для успешного нахождения нулей невязки выбранной численной схемы при фиксированном шаге интегрирования.
    % \end{itemize}
    \begin{itemize}[itemsep=0em]
        \item
            допустимую величину шага интегрирования для заданного алгоритма поиска корней невязки выбранной численной схемы;
        \item
            допустимую степень неявности используемой численной схемы для заданного шага интегрирования и алгоритма поиска корней невязки;
        \item
            сложность алгоритма поиска корней невязки для заданной численной схемы и шага интегрирования,
    \end{itemize}
    при которых возникающая на каждом шаге по времени нелинейная система алгебраических уравнений решается корректно.
\end{definition}

Иначе говоря, нелинейная жёсткость проявляется в необходимости выбирать между неявностью численной схемы,
большим шагом по времени и простотой метода решения нелинейных уравнений.

Таким образом, борьба с нелинейной жёсткостью переводится в плоскость методов оптимизации.
Среди способов улучшения сходимости метода Ньютона можно перечислить линейный поиск \cite{armijo1966minimization, wolfe1969convergence},
метод доверительных областей \cite{sorensen1982newton} и различного рода ускорения \cite{anderson1965iterative, nesterov27method, brown1994convergence}.
Линейный поиск минимизирует невязку вдоль выбранного направления путём подбора оптимального шага.
Метод доверительных областей изменяют направление шага, используя информацию о производных высшего порядка.
Ускоренные методы используют историю шагов при решении задачи оптимизации.
Возможна также комбинация упомянутых методов \cite{brune2015composing}.
Квазиньютоновские методы активно используются для решения уравнений, возникающих при интегрировании жёстких систем
\cite{brown1985experiments, alexander1991modified, moore1994stepsize, schlenkrich2006application}.
Данная группа методов решает задачу оптимизации или поиска корней уравнения, используя аппроксимации производных, а не их точные значения.
Все эти методы отличаются необходимым количеством вычислений невязки, якобиана или гессиана в ходе поиска решения.
