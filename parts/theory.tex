\chapter{Теоретические сведения}
\label{chapter:theory} \index{Теория}

Прежде чем перейти к описанию численных методов, остановимся подробнее на теории жёстких систем дифференциальных уравнений.
В данном разделе приведены необходимые для дальнейшего анализа элементы линейной и нелинейной теории устойчивости численных методов.
Отметим, что данная теория гораздо более полно описана в работах \cite{heirer1999solvingode2, lambert1991methods}.

В дальнейшем часто будет рассматриваться задача Коши вида
%
\begin{equation}
    \label{eq:initial_value_problem}
    \begin{dcases}
        \frac{d \bvec{x}}{d t} = f(t, \bvec{x}), \\
        \bvec{x}(0) = \bvec{x}_0
    \end{dcases}
\end{equation}
%
Поскольку любую систему обыкновенных дифференциальных уравнений можно свести к автономной, добавив уравнение $ dt / dt = 1 $,
запись задачи можно упростить:
%
\begin{equation}
    \label{eq:autonomous_initial_value_problem}
    \begin{dcases}
        \frac{d \bvec{x}}{d t} = f(\bvec{x}), \\
        \bvec{x}(0) = \bvec{x}_0
    \end{dcases}
\end{equation}
%
Раскладывая $ f(\bvec{x}) $ в ряд Тейлора в окрестности $ \bvec{x}_0 $ и пренебрегая членами, содержащими производные порядка выше второго, можно получить линеаризованную задачу Коши:
%
\begin{equation}
    \label{eq:linearized_initial_value_problem}
    \begin{dcases}
        \frac{d \bvec{x}}{d t} = f_0 + F(\bvec{x}_0) \cdot (\bvec{x} - \bvec{x}_0), \\
        \bvec{x}(0) = \bvec{x}_0,
    \end{dcases}
\end{equation}
%
где $ \left. \frac{\partial f}{\partial \bvec{x}} \right|_{\bvec{x}_0} = F(\bvec{x}_0) \equiv F_0 $~--- матрица Якоби правой части уравнения в точке $ \bvec{x}_0 $.

Для линеаризованной задачи \eqref{eq:linearized_initial_value_problem} известно точное решение.
%(в общем случае совпадающее с решением задачи \eqref{eq:autonomous_initial_value_problem} только при условии линейности $ f $ по $ \bvec{x} $).
В случае, если $ F_0 $ обратима, оно имеет вид
\begin{equation}
    \label{eq:linearized_solution}
    \bvec{x}(t) = \bvec{x}_0 + (\exp(t \cdot F_0) - I) F_0^{-1} \cdot f_0
\end{equation}

\section{Линейная теория устойчивости}
\label{sec:linear_stability}

В рамках линейной теории устойчивости рассматривается поведение численного метода при решении уравнения \eqref{eq:linearized_initial_value_problem}.
%В случае $ f_0 = F_0 \cdot \bvec{x}_0 $ задача превращается в \emph{уравнение Далквиста}:
В случае невырожденной $ F_0 $ при помощи линейной замены $ \bvec{x}' = \bvec{x} - \bvec{x}_0 + F_0^{-1} f_0 $ задача сводится к \emph{уравнению Далквиста}:
%
\begin{equation}
    \label{eq:Dahlquist_equation}
    \begin{dcases}
        \dfrac{d \bvec{x}}{d t} = F_0 \cdot \bvec{x}, \\
        \bvec{x}(0) = \bvec{x}_0
    \end{dcases}
    \qquad
    \Longleftrightarrow
    \qquad
    \bvec{x}(t) = \exp(t \cdot F_0) \cdot \bvec{x}_0
\end{equation}

% Один шаг произвольной схемы численного интегрирования, располагающей только значениями правой части, всегда [?] можно записать в виде
% %
% \begin{equation}
%     \label{eq:numerical_step}
%     \bvec{x}^{n+1} = \bvec{x}^n + \phi \cdot \Delta t,
% \end{equation}
% %
% где \emph{наклон} $ \phi $ зависит только от значения правой части $ f(\bvec{x}) $ в некоторых точках.
Для линейных (возможно, многошаговых) численных схем в случае $ f(\bvec{x}) = F_0 \cdot \bvec{x} $ один шаг можно \cite{lambert1991methods} записать в виде
%
\begin{equation}
    \label{eq:stability_function}
    \bvec{x}^{n+1} = R(\Delta t \cdot F_0) \cdot \bvec{x}^n,
\end{equation}
%
где $ R(z) $~--- \emph{функция устойчивости}, а $ R(\Delta t \cdot F_0) $~--- \emph{матрица перехода}.
Обычно $ R(z) $ рассматривается как функция комплексного переменного, и потом естественным образом обобщается на матричный аргумент.
По умолчанию так и будет далее подразумеваться.
Однако в силу специфики данной работы, понятие функции устойчивости будет иногда обобщаться, сразу предполагая матричный характер
(то есть прообраз среди функций комплексного переменного может отсутствовать).

Приведём несколько примеров линейных численных схем и их функций устойчивости.
Эти примеры понадобятся при дальнейшем анализе.
%
\begin{align}
    \text{\emph{Явный метод Эйлера:}}   && \frac{\bvec{x}^{n+1} - \bvec{x}^n}{\Delta t} &= f(\bvec{x}^n) & R(z) &= 1 + z & \label{eq:forward_Euler} \\
    \text{\emph{Неявный метод Эйлера:}} && \frac{\bvec{x}^{n+1} - \bvec{x}^n}{\Delta t} &= f(\bvec{x}^{n+1}) & R(z) &= \frac{1}{1 - z} & \label{eq:backward_Euler} \\
    \text{\emph{Метод трапеций:}}       && \frac{\bvec{x}^{n+1} - \bvec{x}^n}{\Delta t} &= \frac{1}{2} f(\bvec{x}^n) + \frac{1}{2} f(\bvec{x}^{n+1}) & R(z) &= \frac{2 + z}{2 - z} & \label{eq:trapezoid}
\end{align}
%
В общем случае двухточечной схемы:
\begin{align}
    \frac{\bvec{x}^{n+1} - \bvec{x}^n}{\Delta t} &= (1 - M) f(\bvec{x}^n) + M f(\bvec{x}^{n+1}) & R(z) &= (1 - M z)^{-1}(1 + (1 - M) z), \label{eq:weighted_two_point}
\end{align}
где $ M $, вообще говоря, может быть матрицей (в таком случае $ R(z) $ следует изначально рассматривать как матричную функцию).
Если же $ M $~--- число, то данная схема называется \emph{$ \theta $-методом}.


\subsection{Линейная устойчивость}
\label{subsec:linear_stability}

Рассмотрим формально вопросы затухания аналитического и численного решения с течением времени.
Для этого потребуются следующие определения и теоремы:

\begin{definition}
    \label{def:spectral_radius_and_abscissa}
    Пусть $ \sigma(A) \subseteq \CC $~--- спектр линейного оператора $ A $.
    Число $ \displaystyle r(A) = \sup \{|\lambda| \mid \lambda \in \sigma(A) \} $ называется \emph{спектральным радиусом} линейного оператора $ A $,
    а $ \displaystyle s(A) = \sup \{\Re \lambda \mid \lambda \in \sigma(A) \} $~--- \emph{спектральной границей}.
\end{definition}

\begin{theorem}[формула Бёрлинга-Гельфанда]
    \label{thm:Beurling-Gelfand_formula}
    Пусть $ A $~--- линейный ограниченный оператор, действующий в банаховом пространстве $ X $ над $ \CC $.
    Тогда $ \displaystyle r(A) = \lim_{n \to \infty} \| A^n \|^{\frac{1}{n}} = \inf_{n \in \NN} \| A^n \|^{\frac{1}{n}} $.
\end{theorem}

\begin{corollary}
    \label{cor:spectral_radius_norm_bounds}
    Пусть $ A $~--- линейный ограниченный оператор, действующий в банаховом пространстве $ X $ над $ \CC $.
    Тогда
    \[
        \forall \varepsilon > 0 \;\; \exists n_0 \in \NN: \; \forall n > n_0 \quad (r(A))^n \leqslant \| A^n \| < (r(A) + \varepsilon)^n
    \]
\end{corollary}

\begin{proof}
    По теореме \ref{thm:Beurling-Gelfand_formula}
    \[
        \forall \varepsilon > 0 \;\; \exists n_0 \in \NN: \; \forall n > n_0 \quad \left| \| A^n \|^{\frac{1}{n}} - r(A) \right| < \varepsilon,
    \]
    причём $ \forall n \in \NN \;\; r(A^n) = (r(A))^n \leqslant \| A^n \| $.
    Тогда
    \[
        \forall \varepsilon > 0 \;\; \exists n_0 \in \NN: \; \forall n > n_0 \quad r(A) \leqslant \| A^n \|^{\frac{1}{n}} < r(A) + \varepsilon,
    \]
    откуда получаем искомое неравенство.
\end{proof}

\begin{theorem}[об отображении спектра; \cite{takesaki2001opalgebras1}, предложение 2.8]
    \label{thm:spectral_mapping_theorem}
    Пусть $ A $~--- линейный ограниченный оператор, действующий в банаховом пространстве $ X $ над $ \CC $,
    $ f $~--- регулярная в окрестности $ \sigma(A) $ функция.
    Тогда
    \begin{equation}
        \label{eq:spectral_mapping_theorem}
        \sigma(f(A)) = f(\sigma(A))
    \end{equation}
\end{theorem}

\begin{lemma}
    \label{lem:operator_exponential_norm_convergence}
    Пусть $ A $~--- линейный ограниченный оператор, действующий в банаховом пространстве $ X $ над $ \CC $. %, причём $ s(A) < 0 $.
    Тогда
    \[
        s(A) = \inf \left \{ \omega \in \RR: \lim_{t \to +\infty} e^{-\omega t} \cdot \| \exp(t \cdot A) \| = 0 \right \}
    \]
\end{lemma}

\begin{proof}
    Обозначим $ T(t) = \exp(t \cdot A) $.
    Тогда
    \[
        T(t) = \exp((n \Delta t + \tau) \cdot A) = \left( T(\Delta t) \right)^n \cdot T(\tau),
    \]
    где $ \Delta t > 0 $, $ n = \lfloor t / \Delta t \rfloor $, $ \tau = t - n \Delta t \in [0;\Delta t) $.
    В силу ограниченности $ A $ имеем $ \| T(t) \| \leqslant e^{t \cdot \| A \|} $.
    Согласно теореме \ref{thm:spectral_mapping_theorem}, $ \sigma(T(t)) = \exp(t \cdot \sigma(A)) $, из чего следует $ r(T(t)) = e^{t \cdot s(A)} $.
    Применяя следствие \ref{cor:spectral_radius_norm_bounds} и учитывая, что
    \[
        C^{-1} = e^{-\Delta t \cdot \| A \|} \leqslant \| T(-\tau) \|^{-1} = \| T(\tau)^{-1} \|^{-1} \leqslant \| T(\tau) \| \leqslant e^{\Delta t \cdot \| A \|} = C,
    \]
    получаем, что $ \forall \varepsilon > 0 \;\; \exists n_0 \in \NN: \; \forall n > n_0 $
    \[
        C^{-1} \cdot e^{n \Delta t \cdot s(A)} \leqslant \| T(t) \| \leqslant C \cdot e^{n \Delta t \cdot (s(A) + \varepsilon)}
    \]
    Наконец, так как $ n \Delta t = t - \tau $, $ \forall \varepsilon > 0 \;\; \exists t_0 > 0: \; \forall t > t_0 $
    \[
        M^{-1} \cdot e^{t \cdot s(A)} \leqslant \| T(t) \| \leqslant M \cdot e^{t \cdot (s(A) + \varepsilon)},
    \]
    где $ M = C \cdot e^{\Delta t \cdot |s(A)|} = e^{\Delta t \cdot (\|A\| + |s(A)|)} $. %\limto{\Delta t}{0} 1 $.
    Отсюда получаем доказываемое утверждение.
\end{proof}

Следствие \ref{cor:spectral_radius_norm_bounds} позволяет на основе данных о спектре линейного оператора $ A $ оценить асимптотику $ \| A^n \| $,
а лемма \ref{lem:operator_exponential_norm_convergence}~--- $ \| \exp(t \cdot A) \| $.

Вернёмся к уравнению Далквиста \eqref{eq:Dahlquist_equation}.
Нас интересуют равномерные оценки на норму численного решения, получаемого заданным методом при заданном постоянном шаге интегрирования.
Для этого введём следующее определение и утверждение:

\begin{definition}
    \label{def:stability_region}
    Множество $ \stabreg = \{ z \in \CC \mid | R(z) | < 1 \} $ называется \emph{областью абсолютной устойчивости} численного метода, обладающего функцией устойчивости $ R(z) $.
    Множество $ \overline{\stabreg} = \{ z \in \CC \mid | R(z) | \leqslant 1 \} $~--- замыкание области абсолютной устойчивости.
\end{definition}

\begin{statement}
    \label{stat:linear_numerical_stability}
    Пусть численное решение уравнения \eqref{eq:Dahlquist_equation} ищется интегрированием с постоянным шагом $ \Delta t $
    при помощи метода, обладающего функцией устойчивости $ R(z) $ и соответсвующей областью абсолютной устойчивости $ \stabreg $.
    Пусть также $ R(z) $ регулярна в окрестности $ \Delta t \cdot \sigma(F_0) $.
    Тогда $ \bvec{x}^n = \left( R(\Delta t \cdot F_0) \right)^n \cdot \bvec{x}_0 $, и выполнено
    \begin{align}
        \Delta t \cdot \sigma(F_0) \subseteq \stabreg \qquad \Longleftrightarrow \qquad & \| \left( R(\Delta t \cdot F_0) \right)^n \| \limto{n}{\infty} 0 \\
        \Delta t \cdot \sigma(F_0) \subseteq \CC \setminus \overline{\stabreg} \qquad \Longrightarrow \qquad & \| \left( R(\Delta t \cdot F_0) \right)^n \| \limto{n}{\infty} \infty
    \end{align}
\end{statement}

\begin{proof}
    В силу \eqref{eq:stability_function} имеем первое утверждение: $ \bvec{x}^n = \left( R(\Delta t \cdot F_0) \right)^n \cdot \bvec{x}_0 $.
    Далее заметим, что по теореме \ref{thm:spectral_mapping_theorem}
    \[
        \sigma\left( R(\Delta t \cdot F_0) \right) = R\left( \Delta t \cdot \sigma(F_0) \right)
    \]
    Отсюда следует, что
    \begin{align}
        \Delta t \cdot \sigma(F_0) \subseteq \stabreg \qquad \Longleftrightarrow \qquad & r\left( R(\Delta t \cdot F_0) \right) < 1 \\
        \Delta t \cdot \sigma(F_0) \subseteq \CC \setminus \overline{\stabreg} \qquad \Longleftrightarrow \qquad & r\left( R(\Delta t \cdot F_0) \right) > 1
    \end{align}
    Наконец, применяя следствие \ref{cor:spectral_radius_norm_bounds}, завершаем доказательство утверждения.
\end{proof}

В теории линейной устойчивости важная роль отводится \emph{A-устойчивости}~--- свойству численного решения уравнения Далквиста не возрастать по норме,
если не возрастает норма истинного решения.
Если к тому же при увеличении шага интегрирования норма численного решения на следующей итерации (или, быть может, через некоторе заранее известное число итераций)
также стремится к нулю, то говорят об \emph{L-устойчивости}.
Дадим формальное определение.

\begin{definition}
    \label{def:A_stability}
    Численный метод называется \emph{A-устойчивым} в случае, если $ \CC^- \equiv \{ z \in \CC \mid \Re z < 0 \} \subseteq \stabreg $.
\end{definition}

\begin{definition}
    \label{def:L_stability}
    Численный метод называется \emph{L-устойчивым} в случае, если он A-устойчив и выполнено $ \displaystyle\lim_{\Re z \to -\infty} R(z) = 0 $.
\end{definition}

\begin{statement}
    \label{stat:A_stability_criterion}
    Пусть $ R(z) $ регулярна в $ \CC^- $.
    Соответствующий численный метод A-устойчив тогда и только тогда, когда $ \forall F_0 $ выполнено
    \[
        \| \exp(t \cdot F_0) \| \limto{t}{+\infty} 0 \qquad \Longrightarrow \qquad \forall \Delta t > 0 \quad \left \| \left( R(\Delta t \cdot F_0) \right)^n \right\| \limto{n}{\infty} 0
    \]
\end{statement}

\begin{proof}
    В силу леммы \ref{lem:operator_exponential_norm_convergence} импликацию из утверждения можно переписать в виде
    \[
        \sigma(F_0) \subseteq \CC^- \quad \Longrightarrow \quad \forall \Delta t > 0 \quad \left \| \left( R(\Delta t \cdot F_0) \right)^n \right\| \limto{n}{\infty} 0
    \]
    Если также воспользоваться \ref{stat:linear_numerical_stability}, получаем
    \[
        \sigma(F_0) \subseteq \CC^- \quad \Longrightarrow \quad \forall \Delta t > 0 \quad \Delta t \cdot \sigma(F_0) \subseteq \stabreg,
    \]
    что при произвольном $ F_0 $ равносильно $ \CC^- \subseteq \stabreg $.
    Это даёт определение \ref{def:A_stability}.
\end{proof}

\begin{statement}
    \label{stat:L_stability_property}
    Пусть $ R(z) $ регулярна в $ \CC^- $.
    Если соответствующий численный метод L-устойчив, то $ \forall F_0 $ выполнено
    \[
        \| \exp(t \cdot F_0) \| \limto{t}{+\infty} 0 \qquad \Longrightarrow \qquad r \left( R(\Delta t \cdot F_0) \right) \limto{\Delta t}{+\infty} 0
    \]
\end{statement}

\begin{proof}
    Аналогично доказательству утверждения \ref{stat:A_stability_criterion}, перепишем импликацию в виде
    \[
        \sigma(F_0) \subseteq \CC^- \quad \Longrightarrow \quad r \left( R(\Delta t \cdot F_0) \right) \limto{\Delta t}{+\infty} 0
    \]
    В силу теоремы \ref{thm:spectral_mapping_theorem} это эквивалентно
    \[
        \sigma(F_0) \subseteq \CC^- \quad \Longrightarrow \quad \sup \{ |\lambda| \mid \lambda \in R(\Delta t \cdot \sigma(F_0)) \} \limto{\Delta t}{+\infty} 0,
    \]
    что при произвольном $ F_0 $ равносильно $ \forall z \in \CC^- \; | R(\Delta t \cdot z) | \limto{\Delta t}{+\infty} 0 $.
    Это верно для L-устойчивых методов по определению \ref{def:L_stability}.
\end{proof}

Стоит отметить, что утверждение \ref{stat:L_stability_property}, в отличие от \ref{stat:A_stability_criterion}, сформулировано в форме признака, а не критерия.
Также в нём получено лишь утверждение о пределе спектрального радиуса, а не нормы.
По сути, это означает, что получаемая для L-устойчивого метода матрица перехода с увеличением размера шага становится <<почти нильпотентной>>.
Если $ F_0 $ диагонализуема, то из стремления к нулю спектрального радиуса матрицы перехода будет автоматически следовать и стремление к нулю её нормы.


\subsection{Линейная жёсткость}
\label{subsec:linear_stiffness}


Как видно из приведённых результатов, спектр матрицы $ F_0 $ может задавать определённые ограничения на шаг интегрирования $ \Delta t $.
Действительно, если $ s(F_0) < 0 $, но численный метод не A-устойчив, полученное данным методом решение может вести себя некорректно при некоторых $ F_0 $ и $ \Delta t $:
если $ \Delta t \cdot \sigma(F_0) \not\subseteq \stabreg $, то численное решение может возрастать, в то время как аналитическое решение убывает.
В тоже время, такая ситуация невозможна независимо от $ \Delta t $ при использовании A-устойчивых методов.
Но A-устойчивость не гарантирует соизмеримую с аналитическим решением скорость затухания численного; возможен даже случай $ \displaystyle \lim_{\Re z \to -\infty} |R(z)| = 1 $,
что приводит к слабо затухающим осцилляциям численного решения вокруг нуля при сравнительно быстром стремлении к нулю истинного решения.
Если требуется рост скорости затухания за конечное число шагов при увеличении $ \Delta t $, следует пользоваться L-устойчивыми методами.

На рисунках \ref{fig:linear_instability_example}, \ref{fig:linear_instability_example_2} проиллюстрировано поведение явного метода Эйлера (не A-ус\-той\-чи\-вый),
метода трапеций (A-устойчивый, но не L-устойчивый) и неявного метода Эйлера (L-устойчивый) при разных значениях $ z = \Delta t \cdot F_0 $ в одномерной задаче Далквиста.

\begin{figure}[ht!]
    \centering
    \begin{gnuplot}[terminal=epslatex, terminaloptions={color dashed size 16cm,6cm}]
        load './gnuplot/common.gp'

        set style increment default
        set style data lines
        set xlabel  '$ t $'
        set xrange  [ 0 : 10 ] noreverse writeback
        set ylabel  '$ x(t) $' #rotate by 0
        set yrange  [ * : * ] noreverse writeback

        set key width -16

        # Параметры.
        z = -1.5
        N = 5                    # Число точек.
        T = 9.0                  # Время интегрирования.
        lamb = z * (N - 1) / T   # Показатель экспоненты.

        load './gnuplot/Dahlquist.gp'

        set xtics 1
        set xzeroaxis lw 3

        plot Trapezoid using (times[$1]):(Trapezoid[$1]) with linespoints t 'метод трапеций' lw 3 ps 2, \
             BackwardEuler using (times[$1]):(BackwardEuler[$1]) with linespoints t 'неявный метод Эйлера' lw 3 ps 2, \
             ForwardEuler using (times[$1]):(ForwardEuler[$1]) with linespoints t 'явный метод Эйлера' lw 3 ps 2, \
             f(x) t 'истинное решение' lw 4 lc 'black'
    \end{gnuplot}

    \caption{Поведение простейших численных методов при решении одномерного уравнения Далквиста ($ z = -1.5 $)}
    \label{fig:linear_instability_example}
\end{figure}

\begin{figure}[ht!]
    \centering
    \begin{gnuplot}[terminal=epslatex, terminaloptions={color dashed size 16cm,6cm}]
        load './gnuplot/common.gp'

        set style increment default
        set style data lines
        set xlabel  '$ t $'
        set xrange  [ 0 : 10 ] noreverse writeback
        set ylabel  '$ x(t) $' #rotate by 0
        set yrange  [ * : * ] noreverse writeback

        set key width -16

        # Параметры.
        z = -15.0
        N = 5                    # Число точек.
        T = 9.0                  # Время интегрирования.
        lamb = z * (N - 1) / T   # Показатель экспоненты.

        load './gnuplot/Dahlquist.gp'

        set xtics 1
        set xzeroaxis lw 3

        plot Trapezoid using (times[$1]):(Trapezoid[$1]) with linespoints t 'метод трапеций' lw 3 ps 2, \
             BackwardEuler using (times[$1]):(BackwardEuler[$1]) with linespoints t 'неявный метод Эйлера' lw 3 ps 2, \
             f(x) t 'истинное решение' lw 4 lc 'black'
    \end{gnuplot}

    \caption{Поведение простейших численных методов при решении одномерного уравнения Далквиста ($ z = -15 $)}
    \label{fig:linear_instability_example_2}
\end{figure}

Зачастую область устойчивости не A-устойчивых методов ограничена (в частности, это верно для всех явных линейных численных методов) или содержит лишь некоторый подсектор $ \CC^- $,
поэтому ограничение на шаг интегрирования оказывается ограничением сверху.
Таким образом, спектральные свойства $ F_0 $ обуславливают максимально допустимый шаг численного интегрирования.
Ситуацию осложняет следующая теорема, требующая в случае одностадийных схем делать выбор между устойчивостью и высоким порядком аппроксимации:

\begin{theorem}[второй барьер Далквиста]
    \label{thm:Dahlquist_second_barrier}
    Не существует A-устойчивых линейных многошаговых одностадийных схем с порядком аппроксимации выше второго.
\end{theorem}

Ограничение на $ \Delta t $ может сохранятся даже при решении нелинейных задач вида \eqref{eq:autonomous_initial_value_problem}.
Рассмотрим случай, когда характерное время изменения $ F = \frac{\partial f}{\partial \bvec{x}}(\bvec{x}(t)) $
много больше $ \tau = 1/r(F) $~--- характерного времени реакции системы на небольшие возмущения.
В таком случае линеаризация \eqref{eq:linearized_initial_value_problem} остаётся достаточно точной дольше характерного времени $ \tau $.
Это автоматически оставляет в силе ограничения на шаг интегрирования, полученные для линейных систем.
В частности, если $ s(F) < 0 $, но $ \Delta t \cdot \sigma(F) \not\subseteq \stabreg $, численное решение может вести себя неустойчиво к небольшим возмущениям,
в то время как истинное решение, наоборот, будет обладать эффектом демпфирования.

Приведённые выше рассуждения показывают, что определённые системы дифференциальных уравнений могут обладать свойствами,
вынуждающими использовать малый шаг интегрирования при их решении недостаточно устойчивыми в смысле \ref{def:A_stability} и \ref{def:L_stability} методами.
Традиционно такие системы называются \emph{жёсткими}.
Как указано в \cite{heirer1999solvingode2, lambert1991methods}, существует несколько определений жёсткости,
каждое из которых обладает своими достоинствами и недостатками.
Одно из наиболее популярных определений звучит следующим образом:

\begin{definition}
    \label{def:stiffness}
    Система вида $ \frac{\partial \bvec{x}}{\partial t} = f(t, \bvec{x}) $ называется \emph{жёсткой} в том случае,
    если для получения корректного численного решения необходимо использовать шаг интегрирования,
    много меньший характерных масштабов времени, на которых меняется истинное решение.
\end{definition}

Данное определение слишком общее и не отвечает на вопросы о природе ограничения на шаг интегрирования.
На основе всего вышеизложенного анализа мы дадим более узкое, но в некоторой степени и более информативное определение жёсткости.

\begin{definition}
    \label{def:linear_stiffness}
    Система вида $ \frac{\partial \bvec{x}}{\partial t} = f(\bvec{x}) $ называется \emph{линейно жёсткой} в том случае,
    если характерное время (линейной) реакции системы на небольшие возмущения $ \tau = 1 / r \! \left( \frac{\partial f}{\partial \bvec{x}} \right) = 1 / r(F) $
    много меньше характерных масштабов времени, на которых меняется истинное решение.
\end{definition}

Для численных методов с ограниченной областью устойчивости условие $ \Delta t \cdot \sigma(F) \subseteq \stabreg $ влечет $ \Delta t \sim \tau $.
Тогда \ref{def:linear_stiffness} оказывается частным случаем \ref{def:stiffness},
причём необходимость выбора малого шага оказывается обусловленной <<жёстким>> линейным поведением системы в окрестности истинных решений.


\subsection{Методы Рунге-Кутты}
\label{subsec:Runge-Kutta}

Отдельно стоит упомянуть подкласс одношаговых многостадийных линейных численных схем~--- \emph{методы Рунге-Кутты}.
Строгое определение данных методов можно найти в работах \cite{lambert1991methods, heirer1999solvingode2}.
Ограничимся лишь формальным их определением для автономных систем через \emph{таблицы Бутчера}:
\begin{definition}
    Методом Рунге-Кутты с числом стадий $ s $ называется метод вида
    \[
        \bvec{x}^{n+1} = \bvec{x}^n + \Delta t \cdot \sum_{i = 1}^s b_i \cdot \bvec{k}_i
    \]
    \[
        \bvec{k}_i = f\left( \bvec{x}^n + \Delta t \cdot \sum_{j = 1}^s a_{ij} \cdot \bvec{k}_j \right),
    \]
    где вектор $ b = (b_i) $ и матрица $ A = (a_{ij}) $ образуют таблицу Бутчера:
    \[
        \def\arraystretch{1.5}
        \left[
            \begin{array}{c}
                A \\
                \hline
                b
            \end{array}
        \right]
        \quad
        =
        \quad
        \begin{matrix}
            a_{11} & a_{12} & \cdots & a_{1,s}   \\
            a_{21} & a_{22} & \cdots & a_{1,s}   \\
            \cdots & \cdots & \ddots & \vdots    \\
            a_{s,1} & a_{s,2} & \cdots & a_{s,s} \\
            \hline
            b_{1} & b_{2} & \cdots & b_{s} \\
        \end{matrix}
    \]
\end{definition}

Справедлива \cite{heirer1999solvingode2} следующая теорема:
\begin{theorem}
    Если метод Рунге-Кутты имеет порядок аппроксимации $ p $, то
    \[
        R(z) = 1 + z + \frac{z^2}{2!} + \ldots + \frac{z^p}{p!} + \Oclass(z^{p+1}) = e^z + \Oclass(z^{p+1})
    \]
\end{theorem}

\begin{statement}
    Функцию устойчивости произвольного метода Рунге-Кутты можно найти по формуле
    \begin{equation}
        \label{eq:Runge-Kutta}
        R(z) = 1 + z \cdot b^T (I - z \cdot A)^{-1} e = \frac{\det \left( I - z \cdot A + z \cdot e b^T \right)}{\det \left( I - z \cdot A \right)},
    \end{equation}
    где $ e = (1, 1, \ldots, 1)^T $.
\end{statement}
