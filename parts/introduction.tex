\chapter{Введение}
\label{chapter:introduction} \index{Введение}

Написание данной работы мотивировано необходимостью решать жёсткие нелинейные уравнения
реакции при моделировании образования тромба в придатке предсердия левого желудочка.
%Моделирование заключается в решении системы переноса-диффузии-реакции в геометрии левого предсердия или только его придатка.
В качестве реакционной части выступает существенно упрощённая модель каскада свёртывания крови, описываемая девятью компонентами \cite{bouchnita2020mathematical}.
Каскад свёртывания крови~--- это жёсткая система дифференциальных уравнений, имеющая пороговый отклик на изменение параметров модели \cite{shen2008threshold}.
Это вынуждает использовать малый шаг по времени при неявном численном интегрировании \cite{douglas1967generalizedrk}.
В свою очередь, это приводит к непрактично большому времени расчёта упомянутой модели.
С целью увеличения допустимого шага по времени в работе \cite{vassilevski2020parallel} были предложены некоторые модификации реакционной части модели.
Данные модификации заключались в замене нескольких компонент, дающих большой вклад во внедиагональную часть матрицы Якоби, на их экстраполированные значения.
В данной работе мы фокусируемся исключительно на реакционной части модели, исследуя некоторые обобщения описанного выше метода.
Предложенный в настоящей работе численный метод основан на замечании, что матрица Якоби правой части уравнения реакции обладает седловой структурой,
которая проявляет себя при численном интегрировании с большим шагом по времени.
Мы рассматриваем одношаговые схемы интегрирования по времени, чтобы в дальнейшем встроить полученный метод в полностью неявный интегратор уравнений переноса-диффузии.

Согласно теореме Канторовича \cite{kantorovich1949method,ortega2000iterative}, для липшицевых в окрестности корня функций метод Ньютона локально сходится с квадратичной скоростью.
В случае уравнений, возникающих при решении жёстких систем, односторонняя константа Липшица может оказаться сколь угодно большой, и сходимость ухудшается \cite{auzinger1990note, auzinger1993modern}.
Среди методов по улучшению сходимости метода Ньютона можно перечислить линейный поиск \cite{armijo1966minimization, wolfe1969convergence},
метод доверительных областей \cite{sorensen1982newton} и методы ускорения \cite{anderson1965iterative, nesterov27method, brown1994convergence}.
Линейный поиск минимизирует невязку вдоль выбранного направления путём подбора оптимального шага.
Метод доверительных областей изменяют направление шага, используя информацию о производных высшего порядка.
Методы ускорения используют историю шагов при решении задачи оптимизации.
Возможна также комбинация упомянутых методов \cite{brune2015composing}.
Квазиньютоновские методы активно используются для решения уравнений, возникающих при интегрировании жёстких систем
\cite{brown1985experiments, alexander1991modified, moore1994stepsize, schlenkrich2006application}.
Данная группа методов решает задачу оптимизации или поиска корней уравнения, используя аппроксимации производных, а не их точные значения.
Все эти методы отличаются необходимым количеством вычислений невязки, якобиана или гессиана в ходе поиска решения.
Первый подход заключается в улучшении сходимости метода Ньютона в случае численного интегрирования неявным методом Эйлера.
Предложенный способ можно рассматривать как вариант квазиньютоновского метода, где на каждой итерации используется модифицированная матрица Якоби.
В настоящей работе предложен способ получения модифицированной матрицы Якоби, основанный на решении вспомогательной линеаризованной задачи,
возникающей на каждой ньютоновской итерации и связанной с построением численной схемы специального вида,
соответствующей экспоненциальному интегратору.

Много работ посвящено устойчивости и выбору численных схем \cite{auzinger1993modern, dahlquist1963special, dahlquist1975stability, liu2019study}.
Среди популярных схем можно перечислить метод трапеций, семейство многостадийных методов Рунге-Кутты, формулу дифференцирования назад, методы Розенброка и многие другие.
Простейшим методом является явный метод Эйлера, но он имеет малую область абсолютной устойчивости.
Неявный метод Эйлера обладает гораздо большей областью абсолютной устойчивости, однако требует решения нелинейного уравнения на каждом шаге.
Метод трапеций~--- арифметическое среднее между явным и неявным методом Эйлера~--- всё еще достаточно простая схема,
дающая, однако, хорошие результаты для некоторых жёстких систем \cite{auzinger1989asymptotic}.
В настоящей работе строится аналогичная схема, где явная и неявная части комбинируются при помощи весовой матрицы.
Данная матрица зависит от производной правой части системы дифференциальных уравнений и подбирается так, чтобы итоговая схема давала экспоненциальный интегратор.

В работе также поднимается вопрос определения понятия жёсткости системы дифференциальных уравнений
(отдельно обговорим, что будут рассматриваться только корректно поставленные задачи).
С этой целью приведены основные положения линейной и нелинейной теории устойчивости, взятые из работ
\cite{dahlquist1975stability, dahlquist1963special, lambert1991methods, heirer1999solvingode2}.
В частности, рассмотрена обобщённая линейная задача Коши с ограниченным линейным оператором, действующим в банаховом пространстве.
Приведены спектральные признаки устойчивости, а также формально доказан набор утверждений,
связывающих область устойчивости численного метода и асимптотические свойства операторной экспоненты.
Это позволяет ввести понятие линейной жёсткости и связать с ним линейную теорию устойчивости.
Результаты нелинейной теории устойчивости позволяют обобщить это понятие на произвольные системы.
В работе, однако, показано, что проблемы устойчивости интегрирования систем не всегда связаны только с линейной жёсткостью.
Поэтому также предлагается ввести понятие нелинейной жёсткости,
которое можно связать со сложностью оптимизационных задач,
возникающих при использовании неявных численных методов.
