\chapter{Заключение}
\label{chapter:summary} \index{Заключение}

Несмотря на бурное развитие теории устойчивости численных методов во второй половине прошлого века,
некоторые вопросы данной области и по сей день остаются без ответа.
В частности, до сих пор нет удовлетворительного определения жёстких систем,
даже несмотря на то, что необходимость их решения возникает повсеместно.
Например, уравнения, описывающие химические реакции, могут быть особенно жёсткими в силу разных временных масштабов протекающих процессов,
а также в силу значительной нелинейности правой части.
Возможность численно решать подобные системы в условиях ограниченных вычислительных ресурсов
требует развития теории жёстких систем дифференциальных уравнений,
а также соответствующих численных методов.

Таким образом, в ходе исследования способов устойчивого численного интегрирования жёсткой системы каскада свёртывания крови
в данной работе были получены следующие результаты:
\begin{enumerate}
    \item
        Систематизированы основные положения теории устойчивости численных методов,
        необходимые для исследования жёсткости систем.
        Введены два новых понятия: \emph{линейная} и \emph{нелинейная жёсткость},
        отражающие разную природу жёсткости различных систем.
    \item
        Показано, что для устранения эффектов линейной жёсткости достаточно использовать
        устойчивые в том или ином смысле численные методы.
    \item
        Продемонстрировано, что понятие нелинейной жёсткости осмысленно и информативно:
        существуют нелинейно жёсткие системы, которые, тем не менее,
        некорректно интегрируются A-устойчивыми, L-устойчивыми методами и экспоненциальными интеграторами.
        Причём сложность интегрирования данных систем обусловлена нелинейностью правых частей.
    \item
        Показано, что проблема нелинейной жёсткости лежит в плоскости методов оптимизации,
        так как нелинейная жёсткость
        %проявляет себя в процессе решения алгебраических уравнений,
        %возникающих в результате применения неявных численных схем.
        существенно влияет на поведение метода решения нелинейных систем,
        приводя к сходимости к нефизичным (паразитическим) корням уравнения.
        На ряде задач показано,
        что модифицируя алгоритм решения нелинейных систем можно найти такой метод,
        который выбирает траекторию к физичному корню.
    \item
        Предложен класс квазиньютоновских модификаций неявного метода Эйлера,
        а также класс соответствующих двухточечных численных схем,
        дающих ту же матрицу Якоби при использовании метода Ньютона.
        Оба класса параметризованы весовой функцией $ \theta(z) $,
        используемой для <<фильтрации>> спектра матрицы Якоби правой части системы.
    \item
        Предложен вариант весовой функции $ \theta(z) $,
        дающей экспоненциальный интегратор.
        Обоснован её выбор.
    \item
        Для предложенной функции проведены численные эксперименты на жёстких системах
        дифференциальных уравнений (в том числе и на системе каскада свёртывания крови),
        показывающие преимущество предложенных методов в сравнении с другими двухточечными методами
        в вопросах борьбы с нежелательными эффектами нелинейной жёсткости.
    % \item
    %     Показан основной недостаток предложенных методов:
    %     большое число ньютоновских итераций, требуемых для поиска нулей невязки.
    %     Данный недостаток вызван квазиньютоновским характером методов.
    %     Возможные пути его устранения: <<заморозка>> весовой матрицы на все или несколько ньютоновских итераций,
    %     а также учет её производных.
\end{enumerate}
